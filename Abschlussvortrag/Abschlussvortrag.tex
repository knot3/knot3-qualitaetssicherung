%% LaTeX-Beamer template for KIT design
%% by Erik Burger, Christian Hammer
%% title picture by Klaus Krogmann
%%
%% version 2.1
%%
%% mostly compatible to KIT corporate design v2.0
%% http://intranet.kit.edu/gestaltungsrichtlinien.php
%%
%% Problems, bugs and comments to
%% burger@kit.edu

\documentclass[18pt]{beamer}

%% SLIDE FORMAT

% use 'beamerthemekit' for standard 4:3 ratio
% for widescreen slides (16:9), use 'beamerthemekitwide'

\usepackage{templates/beamerthemekit}
% \usepackage{templates/beamerthemekitwide}

%% TITLE PICTURE

% if a custom picture is to be used on the title page, copy it into the 'logos'
% directory, in the line below, replace 'mypicture' with the 
% filename (without extension) and uncomment the following line
% (picture proportions: 63 : 20 for standard, 169 : 40 for wide
% *.eps format if you use latex+dvips+ps2pdf, 
% *.jpg/*.png/*.pdf if you use pdflatex)

\titleimage{Knot}

%% TITLE LOGO

% for a custom logo on the front page, copy your file into the 'logos'
% directory, insert the filename in the line below and uncomment it

\titlelogo{knot3logo}

% (*.eps format if you use latex+dvips+ps2pdf,
% *.jpg/*.png/*.pdf if you use pdflatex)

%% TikZ INTEGRATION

% use these packages for PCM symbols and UML classes
% \usepackage{templates/tikzkit}
% \usepackage{templates/tikzuml}

% the presentation starts here

\title[Short title]{Knot$^3$ }
\subtitle{Praxis in der Softwareentwicklung WS 2013/14}
\author{ Tobias Schulz, Maximilian Reuter, Pascal Knodel, Gerd Augsburg, Christina Erler, Daniel Warzel}

\institute{Institut für Betriebs- und Dialogsysteme, Lehrstuhl für Computergrafik}

% Bibliography

\usepackage[citestyle=authoryear,bibstyle=numeric,hyperref,backend=biber]{biblatex}
\addbibresource{templates/example.bib}
\bibhang1em

\begin{document}

% change the following line to "ngerman" for German style date and logos
\selectlanguage{ngerman}

%title page
\begin{frame}
\titlepage
\end{frame}

%table of contents
\begin{frame}{Outline/Gliederung}
\tableofcontents
\end{frame}

\section{Das Spiel}
\subsection{Konzept}
\begin{frame}{Idee}
\begin{itemize}
\item Kreatives Aufbau- und Knobelspiel
\item Geometrisches Objekt verändern (Knoten)
\item Eigene Aufgaben erstellen
\item Einfache und eindeutige Darstellung
\end{itemize}
\end{frame}
\begin{frame}{Pflicht-Vorgaben}
\begin{itemize}
\item übersichtliche Darstellung
\item komplexe 3D-Geometrie mit Verdeckungen soll frustfrei wahrnembar sein
\item intuitive Navigation
\item Selektion und Modifikation von Kanten
\item übergehen unmöglicher Kantenzustände
\item Highscores (in diesem Fall nach Zeit)
\item einfaches Datenaustauschformat
\item mindestens zehn eindeutige Levels mit steigendem Schwierigkeitsgrad
\end{itemize}
\end{frame}

\begin{frame}{Frewillige-Vorgaben}
\begin{itemize}
\item Portierbarkeit auf Linux-Systeme
\item Soundeffekte bei Kantenbewegungen
\item Hintergrundmusik
\end{itemize}
\end{frame}

\subsection{Marktanalyse}
\begin{frame}{Marktanalyse}

\begin{itemize}
\item Kein vergleichbares Spiel auf dem Markt
\item Es gibt bereits Werkzeuge zum generieren von Knoten, aber keines dieser Programme setzt auf ein Spielkonzept.

\end{itemize}
\end{frame}

\subsection{Demonstration}
\begin{frame}{Demonstration}
\begin{center}
\Huge \textbf{Demo}
\end{center}

\end{frame}





\section{Zukunft}
\subsection{Vermarktung}
\begin{frame}{Vermarktung}
\begin{itemize}
\item frei erhältlich und ohne DRM-Schutz
\item Sourcecode verfügbar unter der MIT-Lizenz
\item sonstige Dateien verfügbar unter der Creative-Commons-Lizenz
\item Aktuelle Versionen und Sourcecode sind unter knot3.de verfügbar

\end{itemize}
\end{frame}

\section{Fakten}
\subsection{Statistik}
\begin{frame}{Statistik}
\begin{itemize}
\item 6 Entwickler
\item 27391 Codezeilen (Stand 19.03.14)
\item 183 Klassen
\end{itemize}
\end{frame}

\subsection{Herausforderungen}
\begin{frame} {Herausforderungen}
Einarbeitung und Erlernen von:
\begin{itemize}
\item C\#
\item XNA-Framework / Mono-Framework
\item Programmierung von 3D-Anwendungen
\end{itemize}
\end{frame}


\section{Fragen}
\begin{frame}{Fragen}
\begin{center}
\Huge \textbf{Weitere Fragen?}
\end{center}
\end{frame}



%\appendix
%\beginbackup
%
%\begin{frame}[allowframebreaks]{References}
%\printbibliography
%\end{frame}
%
%\backupend

\end{document}
