% !TeX encoding = UTF-8
%



\newpage



\section{Pflichtenheft}
\label{Abschnitt:Tests:Protokoll:Pflichtenheft-Verweise}

Die Tabelle \ref{Pflichtenheft:Tests:Referenzverweise} ordnet den im Pflichtenheft vorspezifizierten Testfällen einen Verweis in das Testprotokoll - wo alle Tests beschrieben werden - unter Abschnitt \ref{Abschnitt:Tests:Protokoll} zu. Da sich die Beschreibungen beim Nachspezifizieren ändern und weiter aufgliedern, erleichtert die Zuordnung das Auffinden der Pflicht-Untersuchungen. Im PDF-Dokument zum QS-Bericht führt ein Klick auf einen Bezeichner in der Spalte \textbf{Testprotokoll} direkt zu der entsprechenden Stelle im Protokoll. Während  der Testphase geben die Verweise eine ständige Übersicht zum aktuellen Fortschritt und verhindern, dass bei der Vielzahl von Tests etwas vergessen wird.\\

 

\begin{longtable}{p{0.5\hsize}p{0.275\hsize}p{0.275\hsize}}

	\caption{Pflichtenheft-Testf{"a}lle, Referenzverweise\\~\\}
	\label{Pflichtenheft:Tests:Referenzverweise}
	\\

% Spalten-Überschriften:

	    \multirow{2}{*}{\textbf{Testfall}}
	  & \multicolumn{2}{c}{\textbf{Verweis}}
	  
	  \\ [14pt]
	  
	  & \multicolumn{1}{c}{\textbf{Pflichtenheft}}
	  & \multicolumn{1}{c}{\textbf{Testprotokoll}}
	  
	  \\ 
	     
% Spalten-Inhalte:

% Funktionstests

	  \multicolumn{3}{l}{\textbf{Funktionstests:}}
	  
	\\
	
	  \multicolumn{3}{l}{~}
	  
	\\

	  \multicolumn{1}{L{6.5cm}}{Einstellen gültiger Grafikauflösungen}
	& \multicolumn{1}{c}{/PTF\_10/}
	& \multicolumn{1}{c}{\hyperref[FT:1]{FT\_1}}
	
	\\
	
	
	  \multicolumn{1}{L{6.5cm}}{Bedienen der Nutzerschnittstellen}
	& \multicolumn{1}{c}{/PTF\_20/}
	& \multicolumn{1}{c}{\hyperref[]{...}}
		
	\\

	  \multicolumn{1}{L{6.5cm}}{Transformieren von Knoten in gültige Knoten}
	& \multicolumn{1}{C{3cm}}{/PTF\_20/,\newline/PTF\_70/~~}
	& \multicolumn{1}{c}{\hyperref[FT:10]{FT\_10}}
	
	\\
	
	 \multicolumn{1}{L{6.5cm}}{Erstellen von Challenge-Leveln}
	& \multicolumn{1}{c}{/PTF\_30/}
	& \multicolumn{1}{c}{\hyperref[FT:20]{FT\_20}}
	
	\\
	
	  \multicolumn{1}{L{6.5cm}}{Beenden des Programms}
	& \multicolumn{1}{c}{/PTF\_50/}
	& \multicolumn{1}{c}{\hyperref[FT:40]{FT\_40}}
	
	\\
	
	  \multicolumn{1}{L{6.5cm}}{Pausieren und Beenden von Spielen}
	& \multicolumn{1}{c}{/PTF\_60/}
	& \multicolumn{1}{c}{\hyperref[FT:50]{FT\_50}}
	
	\\
	
	  \multicolumn{1}{L{6.5cm}}{Manuelle Positionierung der Kamera}
	& \multicolumn{1}{c}{/PTF\_80/}
	& \multicolumn{1}{c}{\hyperref[FT:80]{FT\_80}}
	
	\\
	
	  \multicolumn{1}{L{6.5cm}}{Bestehen von Challenge-Leveln}
	& \multicolumn{1}{c}{/PTF\_90/}
	& \multicolumn{1}{c}{\hyperref[FT:60]{FT\_60}}
	
	\\
	
	  \multicolumn{1}{L{6.5cm}}{Speichern und Laden von Knoten}
	& \multicolumn{1}{c}{/PTF\_100/}
	& \multicolumn{1}{c}{\hyperref[FT:70]{FT\_70}}
	
	\\
	
	  \multicolumn{1}{L{6.5cm}}{Einrichten und Entfernen des Programms}
	& \multicolumn{1}{C{3cm}}{/PTF\_120/,\newline/PTF\_130/~~}
	& \multicolumn{1}{c}{\hyperref[FT:TEST90]{FT\_90}}
	
	\\
	


\newpage



% Negativtests

	  \multicolumn{3}{l}{\textbf{Negativtests:}}
	  
	\\
	
	  \multicolumn{3}{l}{~}
	  
	\\
	
	  \multicolumn{1}{L{6.5cm}}{Laden nicht gültiger Knoten-Dateien}
	& \multicolumn{1}{c}{/PTF\_500/}
	& \multicolumn{1}{c}{\hyperref[NT:10]{NT\_10}}
	
	\\
	
	  \multicolumn{1}{L{6.5cm}}{Erstellen von Challenge-Leveln aus gleichen Knoten}
	& \multicolumn{1}{c}{/PTF\_510/}
	& \multicolumn{1}{c}{\hyperref[NT:20]{NT\_20}}
	
	\\
	
	  \multicolumn{1}{L{6.5cm}}{Transformieren von Knoten in nicht gültige Knoten}
	& \multicolumn{1}{c}{/PTF\_520/}
	& \multicolumn{1}{c}{\hyperref[NT:30]{NT\_30}}
	
	\\
	
	  \multicolumn{1}{L{6.5cm}}{Löschen von Standard-Leveln}
	& \multicolumn{1}{c}{/PTF\_530/}
	& \multicolumn{1}{c}{\hyperref[NT:40]{NT\_40}}
	
	\\
	
	  \multicolumn{1}{L{6.5cm}}{Verhalten beim Drücken von nicht belegten Tasten}
	& \multicolumn{1}{c}{/PTF\_1020/}
	& \multicolumn{1}{c}{\hyperref[NT:50]{NT\_50}}
	
	\\
	\\
	\\
	
	
	
% Extremtests, Benchmarks

	  \multicolumn{3}{l}{\textbf{Extremtests, Benchmarks:}}
	  
	\\
	
	  \multicolumn{3}{l}{~}
	  
	\\
	
	  \multicolumn{1}{L{6.5cm}}{Laden großer Knoten-Dateien}
	& \multicolumn{1}{c}{/PTF\_1000/}
	& \multicolumn{1}{c}{\hyperref[]{...}}
	
	\\
	
	  \multicolumn{1}{L{6.5cm}}{Erstellen von großen Challenge-Leveln}
	& \multicolumn{1}{c}{/PTF\_1010/}
	& \multicolumn{1}{c}{\hyperref[]{...}}
	
	\\
	
	  \multicolumn{1}{L{6.5cm}}{Erfassen der maximal möglichen Eingabegeschwindigkeit}
	& \multicolumn{1}{c}{/PTF\_1020/}
	& \multicolumn{1}{c}{\hyperref[]{...}}
	
	\\
	
\end{longtable}


