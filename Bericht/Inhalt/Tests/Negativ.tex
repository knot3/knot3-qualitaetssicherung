% !TeX encoding = UTF-8
%



\newpage



\phantomsection
\label{Abschnitt:Tests:Protokoll:Negativ}



\subsection*{Negativtests}

Problematische Eingaben und Spielsituationen werden hier explizit getestet.\\



\begin{description}


% Laden nicht gültiger Knoten-Daten.

\phantomsection
\label{NT:10}

	\item[NT\_10] \textit{Laden nicht gültiger Knoten-Daten.} \hfill\\
	
	Es ist nicht möglich ungültige Knoten-Daten zu laden. Der Algorithmus lässt dies nicht zu. Nur gültige Daten, die dem von uns spezifizierten Format entsprechen werden in der Auswahlliste angezeigt.\\

	
% Erstellen von Challenge-Leveln aus gleichen Knoten.

\phantomsection
\label{NT:20}
	
	\item[NT\_20] \textit{Erstellen von Challenge-Leveln aus gleichen Knoten.} \hfill\\
	
	Das Erstellen einer Challenge mit gleichen Knoten ist nicht möglich. Wählt der Spieler beim Erstellen einer Challenge zwei gleiche Knoten aus, so kann er nicht auf den \glqq Create!\grqq~-Button drücken.\\
	
	
% Transformieren von Knoten in nicht gültige Knoten.

\phantomsection
\label{NT:30}

	\item[NT\_30] \textit{Transformieren von Knoten in nicht gültige Knoten.} \hfill\\
	
	Da nur gültige Transformationen durchführbar sind, ist es nicht möglich einen ungültigen Knoten mit Hilfe von Transformationen durch die dem Spieler zur Verfügung gestellten Eingabemethoden zu erstellen.\\
	

% Löschen von Standard-Leveln.

\phantomsection
\label{NT:40}
	
	\item[NT\_40] \textit{Löschen von Standard-Leveln.} \hfill\\
	
	Löschen von Spielständen wird über das Löschen der Dateien im persönlichen Ordner vorgenommen.
	Unter Windows sind die Standard-Level nicht im persönlichem Ordner und können somit dort auch nicht gelöscht werden.
	Unter Linux werden nicht mehr vorhandene Standard-Level beim Starten neu in den persönlichen Ordner kopiert.\\
	

% Verhalten beim Drücken von nicht belegten Tasten.

\phantomsection
\label{NT:50}	
	
	\item[NT\_50] \textit{Verhalten beim Drücken von nicht belegten Tasten.} \hfill\\

	Drückt der Spieler nicht belegte Tasten so passiert nichts. Das Spiel läuft ohne Probleme weiter. Es verhält sich so in allen Spielsituationen.\\
	
\end{description}


