% !TeX encoding = UTF-8
%



\newpage



\phantomsection
\label{Abschnitt:Tests:Protokoll:Abnahme}



\subsection*{Spielbarkeitstests}

Kritiken und Kommentare von Testspielern.

\subsubsection*{Testperson 1}
Schülerin ohne große Spielerfahrung oder nähe zur Informatik. Das Spielkonzept wurde erst verständlich nachdem Lesen der Spielanleitung. Die grundlegende Steuerung wurde als größten Teils intuitiv empfunden. Also für die Transformation des Knoten war keine große Einarbeitungszeit von Nöten. Besondere Funktionen wie Selektion mehrerer Kanten oder das Einfärben von Kanten wurden von ihr schnell durch das Einstellungsmenü erfasst. Auch das aufrufen des Menüs über Escape war intuitiv klar für die Testerin. Bei den einfacheren Challanges kamen keine Probleme auf. Bei den schwereren hat man erkannt das die Orientierung im Raum etwas mehr Eingewöhnungszeit benötigt. Nach ein paar Minuten waren schwerere Challanges auch nicht mehr problematisch. Während des Testes kamen kleinere Fehler auf die aber den Spielablauf nicht gestört haben und schnell behoben werden konnten.





