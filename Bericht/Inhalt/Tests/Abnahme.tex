% !TeX encoding = UTF-8
%



\newpage



\phantomsection
\label{Abschnitt:Tests:Protokoll:Abnahme}



\subsection*{Spielbarkeitstests}

Kritiken und Kommentare von Testspielern.\\



\subsubsection*{Testperson 1}

Schülerin ohne große Spielerfahrung oder nähe zur Informatik. Das Spielkonzept wurde erst nach dem Lesen der Spielanleitung verständlich. Die grundlegende Steuerung wurde als nicht intuitiv empfunden. Wie Knoten transformiert werden wurde dagegen schnell entdeckt. Besondere Funktionen wie Selektion mehrerer Kanten oder das Einfärben von Kanten schaute sie im Einstellungsmenü nach. Auch das Aufrufen des Menüs über Escape war der Testerin intuitiv klar. Bei den einfacheren Challanges gab es keine Probleme. Bei den schwereren war etwas Einarbeitungszeit für die Orientierung im Raum notwendig. Nach ein paar Minuten waren schwerere Challanges auch nicht mehr problematisch. Während des Tests traten kleinere Fehler auf, die den Spielablauf jedoch nicht weiter störten und schnell behoben werden konnten. Erstens wurde die Drehungen im linken Bereich nicht sofort in den rechten übertragen. Drehungen im rechten Bereich hingegen wurden direkt im linken Bereich übernommen. Der zweite Fehler war, dass bei mehreren Drehungen im linken Bereich die Rotation durch die nicht sofortige Übernahme im rechten Bereich etwas abwich.







