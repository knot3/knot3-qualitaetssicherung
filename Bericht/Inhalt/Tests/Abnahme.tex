% !TeX encoding = UTF-8
%



\newpage



\phantomsection
\label{Abschnitt:Tests:Protokoll:Abnahme}



\subsection*{Spielbarkeitstests}

Kritiken und Kommentare von Testspielern.\\



\subsubsection*{Testperson 1}

Schülerin ohne große Spielerfahrung oder nähe zur Informatik. Das Spielkonzept wurde erst nach dem Lesen der Spielanleitung verständlich. Die grundlegende Steuerung wurde als nicht intuitiv empfunden. Wie Knoten transformiert werden wurde dagegen schnell entdeckt. Besondere Funktionen wie Selektion mehrerer Kanten oder das Einfärben von Kanten schaute sie im Einstellungsmenü nach. Auch das Aufrufen des Menüs über Escape war der Testerin intuitiv klar. Bei den einfacheren Challanges gab es keine Probleme. Bei den schwereren war etwas Einarbeitungszeit für die Orientierung im Raum notwendig. Nach ein paar Minuten waren schwerere Challanges auch nicht mehr problematisch. Während des Tests traten kleinere Fehler auf, die den Spielablauf jedoch nicht weiter störten und schnell behoben werden konnten. Erstens wurde die Drehungen im linken Bereich nicht sofort in den rechten übertragen. Drehungen im rechten Bereich hingegen wurden direkt im linken Bereich übernommen. Das zweite Problem war, dass bei mehreren Drehungen im linken Bereich die Rotation durch die verzögerte Darstellung im rechten Bereich etwas abwich.

\subsubsection*{Testperson 2}

Informatik-Studentin mit Spielerfahrung. Das Design des Spiels wurde als ansprechend empfunden. Die grundlegende Kamerasteuerung wurde erst nach dem Lesen der Spielanleitung verständlich. Die Menüführung wurde als intuitiv empfunden. Wie Knoten transformiert werden wurde schnell verstanden. Bei den einfacheren Challanges gab es keine Probleme. Bei den schwereren war etwas Einarbeitungszeit nötig. Während dem Bearbeiten der schwereren Challenges wurden besondere Funktionen wie die Selektion mehrerer Kanten oder das Einfärben von Kanten relativ schnell begriffen. Im Creative-Mode hatte die Testperson Spaß beim Ausleben ihrer ersten Knotenideen.

\subsubsection*{Testperson 3}

Informatik Student mit Spielerfahrung. Ohne irgendwelche Erklärungen und Anleitungen direkt in die erste Challenge gegangen und sich dann erst überlegt: \glqq Was kann ich jetzt hier machen?\grqq Schnell wurden die Möglichkeiten die Kamera zu drehen gefunden, auch wenn die Richtung in der sie sich bewegt zunächst irritierend war. Nach einer Kurzen Erklärung was das Spielziel ist und wie man eine Transformation durchführt wurde das erste Level geschafft. Das zweite Level benötigte noch den kurzen Hinweis, das die Richtung der Übergänge eine Rolle spielt und das man Kanten über andere hinweg ziehen kann. Die weiteren konnten dann ohne Spezielle Hilfe angegangen werden. Ab den vierten Level wurde es als Herausforderung empfunden. Im weiterem Verlauf wurde der Creative Modus erforscht um für die höheren Level Erfahrungen zu sammeln. Im Allgemeinen wurde das Spiel als ansprechend und gut empfunden.

{\bf Fazit}: Die Schwierigkeit der Level wurde exakt wie beabsichtigt empfunden. Die Anweisungen aus der Anleitung und dem darin enthaltenem Tutorial wären ausreichend gewesen.
Im gesamten wurde es als interessante Idee bezeichnet die man auch weiterhin testen will.




