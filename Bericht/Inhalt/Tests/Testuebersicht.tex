% !TeX encoding = UTF-8
%



\section{{"U}bersicht}
\label{Abschnitt:Tests:Uebersicht}



\subsection{Kategorien}
\label{Abschnitt:Tests:Uebersicht:Kategorien}

Wir gliedern die von uns durchgeführten Testfälle in verschiedene Kategorien:\\



\begin{description}


% Funktionstests

	\item[Funktionstests] \hfill
	\\
	
	In dieser Kategorie testen wir die Spielfunktionen. Dabei handelt es sich um verbale Beschreibungen von Vorgehen, deren Durchführbarkeit wir mehrfach erfolgreich getestet haben. Wir gewährleisten, dass diese aufgelistete Funktionalität durchführbar ist. Das sind Funktionen und Abläufe, welche für die Spielbarkeit von Knot3 benötigt werden. Die Liste steht im entsprechenden Abschnitt im Testprotokoll, ab \hyperref[Abschnitt:Tests:Protokoll:Funktion]{\mousecursor~S. \pageref{Abschnitt:Tests:Protokoll:Funktion}}.\\


% Komponententests
	
	\item[Komponententests] \hfill
	\\
	
	Das sind Tests zu einzelnen C\#-Komponenten unseres Projekts. Da verschiedene Komponententests oft sehr ähnlich in der Umsetzung sind (Einsetzen von Beispiel-objekten/-werten) geben wir im entsprechenden Abschnitt im Testprotokoll, ab \hyperref[Abschnitt:Tests:Protokoll:Komponenten]{\mousecursor~S. \pageref{Abschnitt:Tests:Protokoll:Komponenten}}, eine Zusammenfassung an, anstatt jeden Komponententest einzeln zu beschreiben.
	Zu fast jeder Komponente führen wir Tests durch, außer an reinen Grafik- oder Daten-Komponenten. Die Gründe dafür sind im Abschnitt \hyperref[Abschnitt:Tests:Nicht]{\mousecursor~\ref{Abschnitt:Tests:Nicht}, ab S. \pageref{Abschnitt:Tests:Nicht}} nochmals im Detail erklärt. Die Statistik zur Testabdeckung durch Komponententests ist unter Abschnitt \hyperref[Abschnitt:Tests:Statistik:Abdeckung]{\mousecursor~\ref{Abschnitt:Tests:Statistik:Abdeckung}, ab S. \pageref{Abschnitt:Tests:Statistik:Abdeckung}} verfügbar.\\


\clearpage


% Negativtests

	\item[Negativtests] \hfill
	\\
	
	Die Negativtests prüfen, ob das Spiel eine (falsche) Eingabe oder Bedienung, welche nicht den Anforderungen an die Anwendung entsprechen erwartungsgemäß abweist. Siehe dazu im entsprechenden Abschnitt im Testprotokoll, \hyperref[Abschnitt:Tests:Protokoll:Negativ]{ab \mousecursor~S. \pageref{Abschnitt:Tests:Protokoll:Negativ}}.\\
		

% Extremtests
	
	\item[Extremtests] \hfill
	\\
	
	Bei den Extremtests prüfen wir, ob das Spiel  größere Datenmengen Problemlos verarbeitet und wo die oberen Schranken liegen. Zudem führen wir einfaches Benchmarking durch. Wir messen den Zeitbedarf kritischer Funktionen und schauen nach möglichen Flaschenhälsen. Die Testergebnisse finden sich im entsprechenden Abschnitt im Testprotokoll ab \hyperref[Abschnitt:Tests:Protokoll:Extrem]{\mousecursor~S. \pageref{Abschnitt:Tests:Protokoll:Extrem}}.\\


% Abnahmetests

	\item[Abnahmetests] \hfill
	\\
	
	Hierbei lassen wir menschliche Tester unser Spiel spielen. Deren Kommentare und Kritiken zu Knot3 sind im entsprechenden Abschnitt im Testprotokoll ab \hyperref[Abschnitt:Tests:Protokoll:Abnahme]{\mousecursor~S. \pageref{Abschnitt:Tests:Protokoll:Abnahme}} beschrieben.\\
		
\end{description}







