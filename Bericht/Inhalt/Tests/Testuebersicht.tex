% !TeX encoding = UTF-8
%



\section{{"U}bersicht}
\label{Abschnitt:Tests:Uebersicht}



\subsection{Kategorien}
\label{Abschnitt:Tests:Uebersicht:Kategorien}

Wir gliedern die von uns durchgeführten Testfälle in verschiedene Kategorien:\\


\begin{description} % TODO: Beschreibungen einfügen.

% Funktionstests

	\item[Funktionstests] \hfill
	\\
	
	In dieser Kategorie testen wir die Spielfunktionen. Das sind Funktionen und Abläufe, welche für die Spielbarkeit von Knot3 benötigt werden.\\
	  
% Komponententests
	
	\item[Komponententests] \hfill
	\\
	
	Zu fast jeder Komponente werden Tests durchgeführt, außer Grafik-Komponenten und reine Daten. Die Gründe dafür sind in Abschnitt \ref{Abschnitt:Tests:Nicht} im Detail beschrieben. Zur Strukturierung der Tests spiegeln wir die organisatorische Struktur des Knot3-Projekts, welches den Programmcode enthält. D.h. zu jeder Komponente die wir testen gibt es eine Testklasse im Tests-Projekt. Eine Statistik zur Testabdeckung durch Komponententests ist unter Abschnitt \ref{Abschnitt:Tests:Statistik:Abdeckung} verfügbar.\\

% Negativtests

	\item[Negativtests] \hfill
	\\
	
	Die Negativtests prüfen, ob das Spiel eine (falsche) Eingabe oder Bedienung, welche nicht den Anforderungen an die Anwendung entsprechen erwartungsgemäß abweist. \\
	
\clearpage
	
% Extremtests
	
	\item[Extremtests] \hfill
	\\
	
	Bei den Extremtests prüfen wir, ob das Spiel  größere Datenmengen Problemlos verarbeitet und wo die oberen Schranken liegen. Zudem führen wir einfaches Benchmarking durch, wobei wir den Zeitbedarf kritischer Funktionen messen und nach möglichen Flaschenhälsen Ausschau halten.\\

% Abnahmetests

	\item[Abnahmetests] \hfill
	\\
	
	Hierbei lassen wir menschliche Tester unser Spiel spielen. Deren Kommentare und Kritiken zu Knot3 sind im Abschnitt Abnahmetests beschrieben.\\
		
\end{description}







