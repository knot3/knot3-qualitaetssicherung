% !TeX encoding = UTF-8
%



\section{{"U}bersicht}
\label{Abschnitt:Tests:Uebersicht}



\subsection{Kategorien}
\label{Abschnitt:Tests:Uebersicht:Kategorien}

Wir gliedern die von uns durchgeführten Testfälle in verschiedene Kategorien:\\


\begin{description} % TODO: Beschreibungen einfügen.

% Funktionstests

	\item[Funktionstests] \hfill
	\\
	
	In dieser Kategorie testen wir die Spielfunktionen. Dabei handelt es sich um Beschreibungen von Vorgehen. Wir gewährleisten, dass die in Abschnitt \ref{Abschnitt:Tests:Protokoll:Funktion} aufgelistete Funktionalität durchführbar ist. Das sind Funktionen und Abläufe, welche für die Spielbarkeit von Knot3 benötigt werden.\\
	  
% Komponententests
	
	\item[Komponententests] \hfill
	\\
	
	Tests zu einzelnen C\#-Komponenten unseres Projekts. Da verschiedene Komponententests oft sehr ähnlich in der Durchführung sind (Einsetzen von Beispiel-objekten/-werten) geben wir in Abschnitt \ref{Abschnitt:Tests:Protokoll:Komponenten} eine Zusammenfassung an, anstatt jeden Komponententest einzeln zu beschreiben.
	Zu fast jeder Komponente führen wir Tests durch, außer Grafik-Komponenten und reine Daten. Die Gründe dafür sind in Abschnitt \ref{Abschnitt:Tests:Nicht} nochmals im Detail erklärt. Die Statistik zur Testabdeckung durch Komponententests ist unter Abschnitt \ref{Abschnitt:Tests:Statistik:Abdeckung} verfügbar.\\

% Negativtests

	\item[Negativtests] \hfill
	\\
	
	Die Negativtests prüfen, ob das Spiel eine (falsche) Eingabe oder Bedienung, welche nicht den Anforderungen an die Anwendung entsprechen erwartungsgemäß abweist. \\
	
\clearpage
	
% Extremtests
	
	\item[Extremtests] \hfill
	\\
	
	Bei den Extremtests prüfen wir, ob das Spiel  größere Datenmengen Problemlos verarbeitet und wo die oberen Schranken liegen. Zudem führen wir einfaches Benchmarking durch, wobei wir den Zeitbedarf kritischer Funktionen messen und nach möglichen Flaschenhälsen Ausschau halten. Die Testergebnisse finden sich unter Abschnitt \ref{Abschnitt:Tests:Protokoll:Extrem}.\\

% Abnahmetests

	\item[Abnahmetests] \hfill
	\\
	
	Hierbei lassen wir menschliche Tester unser Spiel spielen. Deren Kommentare und Kritiken zu Knot3 sind im Abschnitt \ref{Abschnitt:Tests:Protokoll:Abnahme} beschrieben.\\
		
\end{description}







