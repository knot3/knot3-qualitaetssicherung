% !TeX encoding = UTF-8
%



\newpage



\section{Nicht getestet}
\label{Abschnitt:Tests:Nicht}

Hier beschreiben wir nicht getestetes Verhalten und begründen im konkreten Fall unsere Entscheidung einen Test nicht durchzuführen.
~\\



% Nicht durchgeführte Funktionstests:

\subsection*{Funktionstests}
\label{Abschnitt:Tests:Nicht:Funktion}

\begin{description}

	\item[FT\_1] \textit{Einstellung der Grafikauflösung.} \hfill\\
	
	\label{FT:1}
	
	Die möglichen Einstellungen werden dynamisch vom Betriebssystem angefordert. D.h. die Werte, welche dem Spieler zur Auswahl stehen sind bereits vom Betriebssystem auf Gültigkeit überprüft worden, siehe  \href{http://msdn.microsoft.com/de-de/library/microsoft.xna.framework.graphics.graphicsadapter.supporteddisplaymodes}{\mousecursor~Microsoft.Xna-.Framework.Graphics.SupportedDisplayModes}.
	~\\
	
\end{description}



% Nicht durchgeführte Komponententests:

\subsection*{Komponententests}
\label{Abschnitt:Tests:Nicht:Komponenten}

Von den Klassen, die wir für das Komponententesten in Betracht gezogen haben, mussten wir diejenigen ausschließen, die für einen entscheidenden Teil ihrer Funktionalität eine oder mehrere Instanzen der Klassen Game, GraphicsDeviceManager, GraphicsDevice oder ContentManager benötigen. Das bedeutet, dass sie Instanzen dieser Klassen entweder im Konstruktur erstellen, im Konstuktor als Paramater erwarten, dass sie teilweise Wrapper um diese Klassen sind oder dass ihre Funktionalität sich auf einige wenige Methoden beschränkt, die mit diesen Instanzen arbeiten.

Dazu gehören einerseits alle von IRenderEffect erbenden Klassen wegen der intensiven Nutzung von GraphicsDevice und GraphicsDeviceManager sowie teilweise Instanzen von Effect-Klassen, die den Zugriff auf Shader kapseln und ebenfalls von GraphicsDevice und ContentManager abhängen.

Andererseits gehören dazu auch die GameModels und davon erbende Klassen, weil diese eine Instanz von Model enthalten, das über einen ContentManager geladen werden muss und damit ein GraphicsDevice benötigen. Auch die InputHandler, deren hauptsächliche Funktion es ist, in bestimmten, eventbasiert aufgerufenen Methoden, Listen von GameModels zu erstellen, sind damit von ContentManager und vom GraphicsDevice abhängig.
~\\



% Nicht durchgeführte Negativtests:

%\subsection*{Negativtests}
%\label{Abschnitt:Tests:Nicht:Negativ}

%~\\



% Nicht durchgeführte Extremtests:

%\subsection*{Extremtests}
%\label{Abschnitt:Tests:Nicht:Extrem}

%~\\



% Nicht durchgeführte Abnahmetests:

%\subsection*{Abnahmetests}
%\label{Abschnitt:Tests:Nicht:Abnahme}

%~\\


