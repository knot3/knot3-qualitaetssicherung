% !TeX encoding = UTF-8
%



\section{Werkzeuge}
\label{Abschnitt:Tests:Werkzeuge}

Zur Testdurchführung nutzen wir einige Werkzeuge. Wichtig bei deren Wahl waren uns diese Kriterien:

\begin{itemize}

	\item Für Studenten kostenlos, da es hier um ein studentisches Projekt geht.
	\item Aktuelle Open-Source-Software.
	\item Lokale Integrierbarkeit in unsere Entwicklungsumgebung.
	\item Verwendbarkeit zusammen mit dem Versions-Management-System Git über GitHub / Online ... % überarbeiten !!!

\end{itemize}

\subsection{Manuelle Unterst{"u}tzung}
\label{Abschnitt:Programmfehler:Werkzeuge:Manuell}

Lokal unter Visual Studio installierte Werkzeuge sind:

\begin{description}

	\item[NUnit]
	
	\item[OpenCover]
	
	\item[ReportGenerator]

\end{description}

Für deren Zusammenspiel ist es nötig ein Skript zu schreiben. Unter Windows übernimmt diese Aufgabe bei uns eine einfache Stapelverarbeitungsdatei (Batch-Datei/.bat-Dateiendung).
Alternativ und unter Voraussetzung weiterer Kenntnisse können spezielle Build-Skripte verwendet werden.\\
Einerseits war es uns wichtig, dass die Werkzeuge lokal bei jedem verfügbar und ausführbar sind. Andererseits haben wir z.B.


auch eine Automatisierung, da z.B. das individuelle Erstellen der Testabdeckungs-Statistiken sich ständig wiederholt und Zeit verschwendet. % Überleitung zu aut. Pr. ...



\subsection{Automatisierte Pr{"u}fung}
\label{Abschnitt:Programmfehler:Werkzeuge:Automatisiert}


\begin{description}

	\item[OpenCover]
	
	\item[Eigener HTML-Generator]
	
	\item[Travis CI]

\end{description}

