% !TeX encoding = UTF-8
%



\section{Werkzeuge}
\label{Abschnitt:Tests:Werkzeuge}

Zur Testdurchführung helfen uns einige Werkzeuge. Wichtig bei deren Wahl waren uns diese Kriterien:

\begin{itemize}

	\item Kostenlos
	\item Aktuell
	\item Open-Source
	\item In Visual Studio integrierbar
	\item Verwendbarkeit mit Git(-Hub)

\end{itemize}


% Noch schlecht geglieder ... TODO

\subsection{Manuelle Unterst{"u}tzung}
\label{Abschnitt:Programmfehler:Werkzeuge:Manuell}

Eine Anleitung über die Integration und Verwendung der Werkzeuge und hilfreiche Links ist auf GitHub im Wiki unseres Projekts verfügbar. Lokal unter Visual Studio installierte Werkzeuge sind:
\\
\\

\begin{description}

	\item[NUnit], \textit{V. 2.6.3}\hfill
	\\
	\\
	NUnit ist ein Framework für Komponententests für alle .NET-Sprachen.
	
	\begin{tabbing}
		Internetseite:
		\= ~ \href {http://www.nunit.org/}
	    	       {http://www.nunit.org/}
	    \\
	\end{tabbing}
	
	
	\item[OpenCover], \textit{V. 4.5.1923}\hfill
	\\
	\\
	OpenCover ermittelt die Testabdeckung unter .NET-Sprachen ab Version 2.0. Wir nutzen es, um die Testabdeckung durch NUnit-Komponententests zu berechnen.
	
	\begin{tabbing}
		Internetseite:
		\= ~ \href {http://opencover.codeplex.com/}
	    	       {http://opencover.codeplex.com/}
	    \\
	\end{tabbing}
	
	
	\item[ReportGenerator], \textit{V. 1.9.1.0}\hfill
	\\
	\\
	ReportGenerator erstellt zu den von OpenCover produzierten XML-Daten einen übersichtlichen Bericht. Es sind verschiedene Formate möglich. Wir erzeugen z.B. eine HTML-Ausgabe des Berichts.
	\begin{tabbing}
			Internetseite:
			\= ~ \href {http://reportgenerator.codeplex.com/}
		    	       {http://reportgenerator.codeplex.com/}
		    \\
	\end{tabbing}

\end{description}

Für die Integration in Visual Studio sind NuGet Pakete für NUnit, OpenCover und ReportGenerator verfügbar.\\
Um die drei Werkzeuge in Visual Studio verwenden zu können, müssen sie zunächst aufeinander abgestimmt werden. Dazu sind  Build-Skripte nötig. Unter Windows übernimmt diese Aufgabe bei uns eine einfache Stapelverarbeitungsdatei (Batch-Datei/.bat-Dateiendung).\\

Einerseits ist es uns wichtig die Werkzeuge lokal bei jedem Entwickler verfügbar zu machen. Andererseits ist die individuelle Erstellung und Ausführung von Tests alleine noch zu zeitaufwendig. 



\subsection{Automatisierte Tests}
\label{Abschnitt:Tests:Werkzeuge:Automatisiert}

Zusätzlich verwenden wir serverseitige, automatisierte Dienste für Testdurchläufe und die Erstellung von Berichten, welche so ständig auf den neuesten Stand gebracht werden. Die Ergebnisse sind online abrufbar (s. u., ). Über bestandene und fehlgeschlagene Tests werden zudem durch einen Benachrichtigungsservice bei jeder Änderung E-Mails an die Entwickler versandt.
\\
\\


\begin{description}

	\item[OpenCover und ein eigener HTML-Generator] \hfill
	\\
	
	Während unser Projekt läuft ist der automatisch erstellte Bericht über die Testabdeckung unter der Internetadresse
	
	\begin{tabbing}
			\= ~ \href {http://www.knot3.de/development/coverage.php}
					   {http://www.knot3.de/development/coverage.php}
					   
	\end{tabbing} erreichbar.
	\\
	
	\item[Travis Continuous Integration (TCI)] \hfill
	\\
	
	Für private GitHub-Repositories gibt es mit TCI die Möglichkeit nach jedem Commit Tests laufen zu lassen.
	Führt eine Änderung zu Fehlern in bereits vorhandenen Testfällen wird dies in einer E-Mail über die Testzustände nach dem Commit an den Entwickler mitgeteilt. Der Verlauf von fehlerfreien und fehlerhafter Commits ist während der Laufzeit des Projekts unter
	
	\begin{tabbing}
			\= ~ \href {https://travis-ci.org/pse-knot/knot3-code/builds}
		    {https://travis-ci.org/pse-knot/knot3-code/builds}
		   
	\end{tabbing} abrufbar.
	\\
	

\end{description}


