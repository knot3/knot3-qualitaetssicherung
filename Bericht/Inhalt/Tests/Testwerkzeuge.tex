% !TeX encoding = UTF-8
%



\section{Werkzeuge}
\label{Abschnitt:Tests:Werkzeuge}

Zur Unterstützung bei der Testdurchführung nutzten wir zusätzliche Open-Source Werkzeuge, die sich in unsere Entwicklungsumgebung lokal integrieren ließen.\\


\begin{description}

	\item[NUnit]
	
	\item[OpenCover]
	
	\item[ReportGenerator]

\end{description}

Für deren Zusammenspiel war es nötig ein Skript zu schreiben. Unter Windows übernimmt diese Aufgabe bei uns eine einfache Stapelverarbeitungsdatei (Batch-Datei/.bat-Dateiendung).
Alternativ und unter Voraussetzung weiterer Kenntnisse können spezielle Build-Skripte verwendet werden.\\
Einerseits war es uns wichtig, dass die Werkzeuge lokal bei jedem verfügbar und ausführbar sind. Andererseits haben wir z.B.


auch eine Automatisierung, da z.B. das individuelle Erstellen der Testabdeckungs-Statistiken sich ständig wiederholt und Zeit verschwendet. 






