% !TeX encoding = UTF-8
%



\subsection{Funktionstests}
\label{Abschnitt:Tests:Protokoll:Funktion}


\begin{description}

	\item[FT\_001] \textit{Einstellung der Grafikauflösung.} \hfill\\
	
	Die möglichen Einstellungen werden dynamisch vom Betriebssystem angefordert. D.h. die Werte, welche dem Spieler zur Auswahl stehen sind bereits vom Betriebssystem auf Gültigkeit überprüft worden (siehe:  \href{http://msdn.microsoft.com/de-de/library/microsoft.xna.framework.graphics.graphicsadapter.supporteddisplaymodes}{Microsoft.Xna-.Framework.Graphics.SupportedDisplayModes}).\\
	
	\item[FT\_010] \textit{Gültige Knoten-Transformationen.} \hfill\\
	
	\begin{itemize}
	
		\item Creative-Mode startbar.
		\item Startknoten ist sichtbar.
	
		\item Wir definieren eine Liste möglicher Transformationsfolgen ausgehend vom Startknoten. Jeder Punkt ist einzeln ausführbar: \hfill\\

		\begin{enumerate}
	
			\item Jede einzelne Kante des Startknotens ist selektierbar.
			\item Jede einzelne Kante des Startknotens ist in jede Richtung des dreidimensionalen Raumes um einen Schritt durch direktes anklicken und anschließendes Ziehen mit der Maus verschiebbar.
			\item Jede einzelne Kante des Startknotens ist in jede Richtung des dreidimensionalen Raumes um mehrere (mindestens zehn) Schritte durch direktes anklicken und anschließendes Ziehen mit der Maus verschiebbar.
			\item Mehrere (mindestens zwei) selektierte Kanten sind um einen Schritt durch direktes anklicken und anschließendes Ziehen mit der Maus verschiebbar.
			\item Mehrere (mindestens zwei) selektierte Kanten sind um mehrere (mindestens zehn) Schritte durch direktes anklicken und anschließendes Ziehen mit der Maus verschiebbar.
			\item Der in \ref{} abgebildete, Knoten \glqq Schlaufe\grqq ist erstellbar.
			\item Der in \ref{} abgebildete, Knoten \glqq Überleger\grqq ist erstellbar.
			\item Jede einzelne Kante des Startknotens lässt sich nach ihrer Verschiebung wieder in die vorige Position durch direktes anklicken und anschließendes Ziehen zurücksetzen.
			\\
	
		\end{enumerate}
		
		\item Ausgehend von einem Knoten mit mindestens 100 Kanten. \hfill\\
		
	\end{itemize}

\end{description}









