% !TeX encoding = UTF-8
%



\newpage



\subsection*{Funktionstests}
\label{Abschnitt:Tests:Protokoll:Funktion}



\begin{description}

	\item[FT\_1] \textit{Einstellung der Grafikauflösung.} \hfill\\
	
	\label{FT:1}
	
	Die möglichen Einstellungen werden dynamisch vom Betriebssystem angefordert. D.h. die Werte, welche dem Spieler zur Auswahl stehen sind bereits vom Betriebssystem auf Gültigkeit überprüft worden (siehe:  \href{http://msdn.microsoft.com/de-de/library/microsoft.xna.framework.graphics.graphicsadapter.supporteddisplaymodes}{Microsoft.Xna-.Framework.Graphics.SupportedDisplayModes}).
	~\\
	
% 
	
	\item[FT\_10] \textit{Gültige Knoten-Transformationen.} \hfill\\
	
	\label{FT:10}
	
	Wir definieren eine Liste möglicher Transformationen ausgehend vom Startknoten. Jede Transformation ist einzeln ausführbar.\\

	\begin{enumerate}
	
		% Selektieren
		\item Jede einzelne Kante des Startknotens ist selektierbar.
		\item Mehrere Kanten (zwei, drei oder vier) des Startknotens sind selektierbar.
			
		% Direktes Anklicken
		\item Jede einzelne Kante des Startknotens ist in jede Richtung des dreidimensionalen Raumes um einen Schritt durch direktes Anklicken und anschließendes Ziehen mit der Maus verschiebbar.
		\item Jede einzelne Kante des Startknotens ist in jede Richtung des dreidimensionalen Raumes um mehrere (mindestens zehn) Schritte durch direktes Anklicken und anschließendes Ziehen mit der Maus verschiebbar.
		\item Mehrere (mindestens zwei) selektierte Kanten sind um einen Schritt durch direktes Anklicken und anschließendes Ziehen mit der Maus verschiebbar.
		\item Mehrere (mindestens zwei) selektierte Kanten sind um mehrere (mindestens zehn) Schritte durch direktes Anklicken und anschließendes Ziehen mit der Maus verschiebbar.
		
		% Navigationspfeile
		\item Jede einzelne Kante des Startknotens ist in jede Richtung des dreidimensionalen Raumes um einen Schritt durch Anklicken der Navigationspfeile und anschließendes Ziehen mit der Maus verschiebbar.
		\item Jede einzelne Kante des Startknotens ist in jede Richtung des dreidimensionalen Raumes um mehrere (mindestens zehn) Schritte durch Anklicken der Navigationspfeile und anschließendes Ziehen mit der Maus verschiebbar.
		\item Mehrere (mindestens zwei) selektierte Kanten sind um einen Schritt durch Anklicken der Navigationspfeile und anschließendes Ziehen mit der Maus verschiebbar.
		\item Mehrere (mindestens zwei) selektierte Kanten sind um mehrere (mindestens zehn) Schritte durch Anklicken der Navigationspfeile und anschließendes Ziehen mit der Maus verschiebbar.
		
		% Zurücksetzen.
		\item Jede einzelne Kante des Startknotens lässt sich nach ihrer Verschiebung in die vorige Position durch direktes Anklicken und anschließendes Ziehen zurücksetzen.
		\item Jede einzelne Kante des Startknotens lässt sich nach ihrer Verschiebung in die vorige Position durch Anklicken des \glqq Undo\grqq-Buttons zurücksetzen.
		\item Jede einzelne Kante des Startknotens lässt sich nach ihrer Verschiebung in die vorige Position durch Anklicken des \glqq Undo\grqq-Buttons zurücksetzen und der \glqq Redo\grqq-Button macht die Aktion des \glqq Undo\grqq-Buttons rückgängig.
		~\\
	
	\end{enumerate}
	
	% Ausgehend von einem zufällig erstellten Knoten ...
	
	\item[FT\_20] \textit{Erstellen von Challenge-Leveln} \hfill\\
	
	\label{FT:20}
	
	\begin{enumerate} % überarbeiten
	
		\item Im Hauptmenü auf den Text \glqq NEW Creative\grqq~klicken.
		\item Im Creative-Menü auf den Text \glqq NEW Challenge\grqq~klicken.
		\item Im Challenge-Konstruktions-Menü in der linken Liste einen Zielknoten auswählen.
		\item In der rechten Liste einen Startknoten auswählen.
		\item Im Eingabefeld einen Namen für die Challenge eingeben und mit der \glqq ENTER\grqq-Taste bestätigen.
		~\\
		
	\end{enumerate}
	
% Nachbaubare Knoten
	
	\item[FT\_30] \textit{Nachbaubare Knoten-Beispiele} \hfill\\
	
	\label{FT:30}
	
	Eine Sammlung von Beispiel-Knoten verschiedener Komplexität, die nachbaubar sind.
	
	\begin{enumerate}
	
		\label{FT:30:Ueberleger}
		\item \mousecursor~\hyperref[Abb:Test-Knoten:Ueberleger]{\glqq Überleger\grqq}
		
		\label{FT:30:Schlaufe}
		\item \mousecursor~\hyperref[Abb:Test-Knoten:Schlaufe]{\glqq Schlaufe\grqq} 

	\end{enumerate}
	
	
	\item[FT\_40] \textit{Beenden des Programms.} \hfill\\
	
	\label{FT:40}
	
	Durch Klicken des \glqq Exit\grqq~-Buttons im Hauptmenü werden alle laufenden Prozesse des Spiels beendet und der Speicher wird wieder freigegeben.	
	
	\item[FT\_50] \textit{Pausieren und Beenden von Spielen.}  \hfill\\
	
	\label{FT:50}
	
	In beiden Spielmodi besteht die Möglichkeit ein Spiel zu pausieren und zu beenden. Vorgehen beim Pausieren eines laufenden Spiels:

	\begin{enumerate} 
	
		\item Klicken der \glqq ESC\grqq~-Taste im laufendem Spiel öffnet das Pausemenü. Im Challenge-Mode wird die Challenge-Zeit hierbei pausiert.
		\item Im Pausemenü auf den Text  \glqq Back to Game\grqq~ klicken. Hierbei wird das Pausemenü geschlossen und das Spiel fortgesetzt. Im Challenge-Mode läuft nach dem Schließen des Pausemenüs die Challenge-Zeit weiter.
	\end{enumerate}
	
	Vorgehen beim Beenden eines laufenden Spiels im Challenge-Mode:

	\begin{enumerate} 
	
		\item Klicken der \glqq ESC\grqq~-Taste im laufendem Spiel öffnet das Pausemenü.
		\item Im Pausemenü auf den Text  \glqq Abort Challenge\grqq~klicken.
		\item Das Programm schließt die laufende Challenge und öffnet das Hauptmenü.
	\end{enumerate}
	
		Vorgehen beim Beenden eines laufenden Spiels im Creative-Mode:

	\begin{enumerate} 
	
		\item Klicken der \glqq ESC\grqq~-Taste im laufendem Spiel öffnet das Pausemenü.
		\item Im Pausemenü auf den Text \glqq Save and Exit\grqq~ oder \glqq Discard Changes and Exit\grqq~ klicken je nachdem, ob man seinen Knoten speichern möchte oder eben nicht.
		\item Das Programm schließt den Creative-Mode und öffnet das Hauptmenü.
		
		
	\end{enumerate}
	
	
	\item[FT\_60] \textit{Bestehen von Challenge-Leveln.} \hfill\\
	
	\label{FT:60}
	
	Nachdem der Spieler die letzte Transformation zur Beendigung der Challenge getätigt hat, reagiert das Spiel folgendermaßen:
	
	\begin{enumerate} 
	
	    \item Die Challenge-Zeit des Spielers wird gestoppt.
		\item Es öffnet sich ein Dialog, welcher den Spieler auffordert seinen Spielernamen einzugeben.
		\item Nachdem der Spieler den Spielernamen mit der \glqq ENTER\grqq~-Taste bestätigt hat, wird die Highscore-Liste geöffnet.
		\item In der Highscore-Liste kann der Spieler die 10 besten Highscore-Einträge sehen. Wenn die Challenge-Zeit des Spielers gereicht hat, besitzt dieser auch einen Highscore-Eintrag.
		\item Mit Hilfe der zwei Button (\glqq Restart challenge\grqq~ und \glqq Return to menu\grqq~) kann der Spieler die Challenge noch einmal spielen oder zum Hauptmenü zurückkehren.
		
		
	\end{enumerate}
	

	\item[FT\_70] \textit{Speichern und Laden von Knoten.} \hfill\\
	
	\label{FT:70}
	
	Hat der Spieler im Creative-Mode einen Knoten erstellt, so kann er diesen abspeichern und wiederum laden. Dazu muss man folgende Dinge tun:
	
		\begin{enumerate} 
	
		\item  Klicken der \glqq ESC\grqq~-Taste, um das Pausemenü zu öffnen.
		\item Im Pausemenü kann man den Knoten auf unterschiedliche Art und Weise speichern:
		 
		\begin{itemize} 
        \item Man klickt auf den Text \glqq Save\grqq~. Daraufhin wird man aufgefordert den Knotennamen einzugeben, welchen man mit der \glqq ENTER\grqq~-Taste bestätigt. Hat der Knoten bereits einen Knotennamen, so wird man nicht mehr aufgefordert diesen einzugeben. Daraufhin wird der Knoten unter dem Knotennamen gespeichert.
        \item Man klickt auf den Text \glqq Save As\grqq~. Daraufhin wird man aufgefordert den Knotennamen einzugeben, welchen man mit der \glqq ENTER\grqq~-Taste bestätigt. Daraufhin wird der Knoten unter dem Knotennamen gespeichert.
        \item Man klickt auf den Text \glqq Save and Exit\grqq~. Daraufhin wird man aufgefordert den Knotennamen einzugeben, welchen man mit der \glqq ENTER\grqq~-Taste bestätigt. Hat der Knoten bereits einen Knotennamen, so wird man nicht mehr aufgefordert diesen einzugeben. Nach der Bestätigung wird der Knoten gespeichert und das Spiel kehrt zurück zum Hauptmenü.	
		
		\end{itemize}
		
		\item Im Hauptmenü auf den Text \glqq Creative\grqq~klicken.
		\item Im Creative-Menü auf den Text \glqq LOAD Knot\grqq~klicken.
		\item Danach kann man aus der linken Liste den zuvor abgespeicherten Knoten auswählen und mit dem \glqq Load\grqq~-Button laden.
		
	\end{enumerate}
	
\end{description}









