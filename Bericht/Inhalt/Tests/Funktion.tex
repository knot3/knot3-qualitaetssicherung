% !TeX encoding = UTF-8
%



\newpage



\phantomsection
\label{Abschnitt:Tests:Protokoll:Funktion}



\subsection*{Funktionstests}



\begin{description}
	
	
% Gültige Knoten-Transformationen.
	
\phantomsection
\label{FT:10}
	
	\item[FT\_10] \textit{Gültige Knoten-Transformationen.} \hfill\\
	
	Wir definieren eine Liste möglicher Transformationen ausgehend vom Startknoten. Jede Transformation ist einzeln ausführbar.\\

	\begin{enumerate}
	
		% Selektieren
		\item Jede einzelne Kante des Startknotens ist selektierbar.
		\item Mehrere Kanten (zwei, drei oder vier) des Startknotens sind selektierbar.
			
		% Direktes Anklicken
		\item Jede einzelne Kante des Startknotens ist in jede Richtung des dreidimensionalen Raumes um einen Schritt durch direktes Anklicken und anschließendes Ziehen mit der Maus verschiebbar.
		\item Jede einzelne Kante des Startknotens ist in jede Richtung des dreidimensionalen Raumes um mehrere (mindestens zehn) Schritte durch direktes Anklicken und anschließendes Ziehen mit der Maus verschiebbar.
		\item Mehrere (mindestens zwei) selektierte Kanten sind um einen Schritt durch direktes Anklicken und anschließendes Ziehen mit der Maus verschiebbar.
		\item Mehrere (mindestens zwei) selektierte Kanten sind um mehrere (mindestens zehn) Schritte durch direktes Anklicken und anschließendes Ziehen mit der Maus verschiebbar.
		
		% Navigationspfeile
		\item Jede einzelne Kante des Startknotens ist in jede Richtung des dreidimensionalen Raumes um einen Schritt durch Anklicken der Navigationspfeile und anschließendes Ziehen mit der Maus verschiebbar.
		\item Jede einzelne Kante des Startknotens ist in jede Richtung des dreidimensionalen Raumes um mehrere (mindestens zehn) Schritte durch Anklicken der Navigationspfeile und anschließendes Ziehen mit der Maus verschiebbar.
		\item Mehrere (mindestens zwei) selektierte Kanten sind um einen Schritt durch Anklicken der Navigationspfeile und anschließendes Ziehen mit der Maus verschiebbar.
		\item Mehrere (mindestens zwei) selektierte Kanten sind um mehrere (mindestens zehn) Schritte durch Anklicken der Navigationspfeile und anschließendes Ziehen mit der Maus verschiebbar.
		
		% Zurücksetzen.
		\item Jede einzelne Kante des Startknotens lässt sich nach ihrer Verschiebung in die vorige Position durch direktes Anklicken und anschließendes Ziehen zurücksetzen.
		\item Jede einzelne Kante des Startknotens lässt sich nach ihrer Verschiebung in die vorige Position durch Anklicken des \glqq Undo\grqq-Buttons zurücksetzen.
		
		\item[...]
		
\clearpage
		
		\item Jede einzelne Kante des Startknotens lässt sich nach ihrer Verschiebung in die vorige Position durch Anklicken des \glqq Undo\grqq-Buttons zurücksetzen und der \glqq Redo\grqq-Button macht die Aktion des \glqq Undo\grqq-Buttons rückgängig.
		~\\
	
	\end{enumerate}


% Ausgehend von einem zufällig erstellten Knoten ...
% Erstellen von Challenge-Leveln

\phantomsection
\label{FT:20}

	\item[FT\_20] \textit{Erstellen von Challenge-Leveln.} \hfill\\
	
	\begin{enumerate}
	
		\item Durch einen Klick auf den Text \glqq Creative\grqq~öffnet sich das Creative-Menü.
		\item Durch einen Klick auf den Text \glqq NEW Challenge\grqq~im Creative-Menü öffnet sich das Challenge-Konstruktions-Menü.
		\item In der linken Liste einen Startknoten auswählen.
		\item Im Challenge-Konstruktions-Menü in der rechten Liste einen Zielknoten auswählen.
		\item Im Eingabefeld einen Namen für die Challenge eingeben und mit der \glqq ENTER\grqq-Taste bestätigen.
		~\\
		
	\end{enumerate}


% Nachbaubare Knoten-Beispiele.
	
\phantomsection
\label{FT:30}
	
	\item[FT\_30] \textit{Nachbaubare Knoten-Beispiele.} \hfill\\
	
	Eine Sammlung von Beispiel-Knoten zum Nachbauen. Jeder Knoten deckt einen im Spielverlauf immer wieder-auftretenden Modellierungsfall einmalig ab. 
	
	\begin{enumerate}
	
		\label{FT:30:Ueberleger}
		\item \mousecursor~\hyperref[Abb:Test-Knoten:Ueberleger]{\glqq Überleger\grqq}
		
		\label{FT:30:Schlaufe}
		\item \mousecursor~\hyperref[Abb:Test-Knoten:Schlaufe]{\glqq Schlaufe\grqq}
		~\\

	\end{enumerate}


% Beenden des Programms
	
\phantomsection
\label{FT:40}
	
	\item[FT\_40] \textit{Beenden des Programms.} \hfill\\
	
	Ein Klick auf den \glqq Exit\grqq-Button im Hauptmenü beendet das Programm.
	~\\

% Pausieren und Beenden von Spielen

\phantomsection
\label{FT:50}

	\item[FT\_50] \textit{Pausieren und Beenden von Spielen.}  \hfill\\
	
	In beiden Spielmodi besteht die Möglichkeit ein Spiel zu pausieren und zu beenden.
	
	Pausieren eines laufenden Spiels:

	\begin{enumerate} 
	
		\item Durch Drücken der \glqq ESC\grqq~-Taste im laufenden Spiel öffnet sich das Pausemenü. Im Challenge-Mode wird die Challenge-Zeit hierbei pausiert.
		
		\item Durch ein Klick auf den Text \glqq Back to Game\grqq~ im Pausemenü wird dieses Menü geschlossen und das Spiel fortgesetzt. Im Challenge-Mode läuft nach dem Schließen des Pausemenüs die Challenge-Zeit weiter.
		
	\end{enumerate}
		
	~\\Beenden eines laufenden Spiels im Challenge-Mode:

	\begin{enumerate} 
	
		\item Durch Drücken der \glqq ESC\grqq~-Taste im laufendem Spiel öffnet sich das Pausemenü.
		
		\item Durch einen Klick auf den Text  \glqq Abort Challenge\grqq~schließt sich die laufende Challenge und öffnet das Hauptmenü.
		
	\end{enumerate}
	
		~\\Beenden eines laufenden Spiels im Creative-Mode:

	\begin{enumerate} 
	
		\item Durch Drücken der \glqq ESC\grqq~-Taste im laufendem Spiel öffnet sich das Pausemenü.
		
		\item Durch einen Klick auf den Text \glqq Save and Exit\grqq~wird der Knoten gespeichert:
		
		\begin{itemize}
		
			\item Fall 1: Ist bereits ein Dateiname vorhanden, wird dieser beim Speichern verwendet.
			
			\item Fall 2: Ist noch kein Dateiname vorhanden, öffnet sich der \glqq Save As\grqq-Dialog und fordert den Spieler auf einen Namen einzugeben. Der Knoten wird nach Bestätigen dieses Dialogs gespeichert und der Spieler gelangt ins Hauptmenü.\\
		
		\end{itemize}
		
		
		\item Durch einen Klick auf den Text \glqq Discard Changes and Exit\grqq~gelangt der Spieler ins Hauptmenü.	
		~\\	
		
	\end{enumerate}
	

% Bestehen von Challenge-Leveln
	
\phantomsection
\label{FT:60}
	
	\item[FT\_60] \textit{Bestehen von Challenge-Leveln.} \hfill\\
	
	Nachdem der Spieler die letzte Transformation zur Beendigung der Challenge getätigt hat, reagiert das Spiel folgendermaßen:
	
	\begin{enumerate} 
	
	    \item Die Challenge-Zeit des Spielers wird gestoppt.
	    
		\item Es öffnet sich ein Dialog, welcher den Spieler auffordert seinen Spielernamen einzugeben.
		
		\item Nachdem der Spieler den Spielernamen mit der \glqq ENTER\grqq~-Taste bestätigt hat, wird die Highscore-Liste geöffnet.
		
		\item In der Highscore-Liste kann der Spieler die 10 besten Highscore-Einträge sehen. Wenn die Challenge-Zeit des Spielers gereicht hat, besitzt dieser auch einen Highscore-Eintrag.
		
		\item Mit Hilfe der zwei Buttons (\glqq Restart challenge\grqq~ und \glqq Return to menu\grqq~) kann der Spieler die Challenge noch einmal spielen oder zum Hauptmenü zurückkehren.
		~\\
			
	\end{enumerate}
	
	
\clearpage


% Speichern und Laden von Knoten

\phantomsection
\label{FT:70}

	\item[FT\_70] \textit{Speichern und Laden von Knoten.} \hfill\\
	
	Hat der Spieler im Creative-Mode einen Knoten erstellt, so kann er diesen abspeichern und wiederum laden. Dazu muss man folgende Dinge tun:
	
		\begin{enumerate} 
	
		\item  Durch Drücken der \glqq ESC\grqq~-Taste öffnet sich das Pausemenü.
		\item Im Pausemenü kann man den Knoten auf unterschiedliche Art und Weise speichern:
		 
		\begin{itemize} 
		
        \item Durch einen Klick auf den Text \glqq Save\grqq~. wird man aufgefordert den Knotennamen einzugeben, welchen man mit der \glqq ENTER\grqq~-Taste bestätigt. Hat der Knoten bereits einen Knotennamen, so wird man nicht mehr aufgefordert diesen einzugeben. Daraufhin wird der Knoten unter diesem Namen gespeichert.
        
        \item Durch einen Klick auf den Text \glqq Save As\grqq~wird man aufgefordert den Knotennamen einzugeben, welchen man mit der \glqq ENTER\grqq~-Taste bestätigt. Daraufhin wird der Knoten unter diesem Namen gespeichert.
        
        \item Durch einen Klick auf den Text \glqq Save and Exit\grqq~wird man aufgefordert den Knotennamen einzugeben, welchen man mit der \glqq ENTER\grqq~-Taste bestätigt. Hat der Knoten bereits einen Knotennamen, so wird man nicht mehr aufgefordert diesen einzugeben. Nach der Bestätigung wird der Knoten gespeichert und das Spiel kehrt zurück zum Hauptmenü.	
		
		\end{itemize}
		
		\item Im Hauptmenü auf den Text \glqq Creative\grqq~klicken.
		
		\item Durch einen Klick auf den Text \glqq LOAD Knot\grqq~im Creative-Menü öffnet sich das Knoten-Lademenü.
		
		\item Im Knoten-Lademenü kann man aus der linken Liste den zuvor abgespeicherten Knoten auswählen und mit dem \glqq Load\grqq~-Button laden.
		~\\
		
	\end{enumerate}
	
	
% Manuelle Positionierung der Kamera.

\phantomsection
\label{FT:80}
	
	\item[FT\_80] \textit{Manuelle Positionierung der Kamera.} \hfill\\
	
	Mit den folgenden Tastatureingaben kann der Spieler die Kamera manuell bewegen. Die Tastatureingaben sind der Standardtastaturbelegung entnommen (siehe Spielanleitung/Kamerabewegung).
	
	\begin{itemize} 
	
        \item Mit den \glqq WASD\grqq~Tasten bewegt der Spieler die Kamera nach oben/unten/rechts/links.
        
        \item Mit Hilfe der Tasten \glqq R\grqq~ und \glqq F\grqq~ bewegt der Spieler die Kamera in der Ebene nach vorne und hinten.
        
        \item Der Spieler zoomt mit den Tasten \glqq Q\grqq~und \glqq E\grqq~ (alternativ mit dem Mausrad) herein- und heraus.
        
        \item Der Spieler rotiert die Kamera um eine Kante des Knotens, indem er sie mit der rechten Maustaste auswählt und mit den Pfeiltasten (alternativ durch gedrückt halten der rechten Maustaste) um die Kante rotiert.
        
        \item Mittels der \glqq Space\grqq~-Taste springt der Mittelpunkt der Kamera auf den Mittelpunkt der selektierten Kante.
        
        \item Mit der linken \glqq Alt\grqq~-Taste wird die Kamera frei gegeben. Mit der Maus schaut sich der Spieler um. Durch erneutes Klicken der linken \glqq Alt\grqq~-Taste rastet die Kamera wieder ein.
        
        \item Durch drücken der \glqq ENTER\grqq~-Taste setzt der Spieler die Kamera zurück.
        
        ~\\
		
		\end{itemize}
		
		
% Einrichten und Entfernen des Programms

\phantomsection
\label{FT:90}

	\item[FT\_90] \textit{Einrichten und Entfernen des Programms} \hfill\\

	Hinweis: Es gibt in der Endversion von Knot3 keine automatische Installation/Deinstallation, \hyperref[]{}.\\
		
	\begin{enumerate}

		\item Das Archiv, in dem sich alle für das Spiel relevanten Dateien befinden, lässt sich auf dem Zielsystem entpacken.
		
		\item Durch einen Doppel-Klick auf die ausführbare Datei \glqq Knot3.exe\grqq startet das Spiel erstmalig im Fenstermodus und das Hauptmenü wird angezeigt. Dabei wird auf dem Zielsystem auch ein Ordner für Einstellungen und Spielspeicherstände angelegt.
		
		\item Die Deinstallation erfolgt manuell. D.h. alle zu Knot3 gehörigen Ordner sind vom System zu löschen.
		~\\ 
	
	\end{enumerate}
	
	
% Spiel-Modi starten.

\phantomsection
\label{FT:100}
	
	\item[FT\_100] \textit{Spiel-Modi starten} \hfill\\
	
	Creative-Mode:\\
	
	\begin{enumerate}
	
		\item Durch einen Klick auf den Text \glqq Creative\grqq~im Hauptmenü öffnet sich das Creative-Menü.
		
		\item Durch einen Klick auf den Text \glqq NEW Knot\grqq~startet der Creative-Mode zum Erstellen eines neuen Knotens.
		
		\item Als Start-Knoten wird ein Quadrat angezeigt.\\
	
	\end{enumerate}
	
	Challenge-Mode:\\
	
	\begin{enumerate}
	
		\item Durch einen Klick auf den Text \glqq Challenge\grqq~im Hauptmenü öffnet sich das Challenge-Menü.
		
		\item Im Challenge-Menü wird in der Challenge-Liste eine Challenge ausgewählt und durch einen Klick auf den Start-Button gestartet.
		
		\item Auf der linken Seite des Bildschirms wird der Referenzknoten, auf der rechten Seite der zu bearbeitende Knoten angezeigt.
	
	\end{enumerate}
	~\\

	
\end{description}









