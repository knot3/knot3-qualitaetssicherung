% !TeX encoding = UTF-8
%



\subsection{Funktionstests}
\label{Abschnitt:Tests:Protokoll:Funktion}


\begin{description}

	\item[FT\_001] \textit{Einstellung der Grafikauflösung.} \hfill\\
	
	Die möglichen Einstellungen werden dynamisch vom Betriebssystem angefordert. D.h. die Werte, welche dem Spieler zur Auswahl stehen sind bereits vom Betriebssystem auf Gültigkeit überprüft worden (siehe:  \href{http://msdn.microsoft.com/de-de/library/microsoft.xna.framework.graphics.graphicsadapter.supporteddisplaymodes}{Microsoft.Xna-.Framework.Graphics.SupportedDisplayModes}).\\
	
	\item[FT\_010] \textit{Gültige Knoten-Transformationen.} \hfill\\
	
	Wir definieren eine Liste möglicher Transformationen ausgehend vom Startknoten. Jede Transformation ist einzeln ausführbar.\\

	\begin{enumerate}
	
		% Selektieren
		\item Jede einzelne Kante des Startknotens ist selektierbar.
		\item Mehrere Kanten (zwei, drei oder vier) des Startknotens sind selektierbar.
			
		% Direktes Anklicken
		\item Jede einzelne Kante des Startknotens ist in jede Richtung des dreidimensionalen Raumes um einen Schritt durch direktes Anklicken und anschließendes Ziehen mit der Maus verschiebbar.
		\item Jede einzelne Kante des Startknotens ist in jede Richtung des dreidimensionalen Raumes um mehrere (mindestens zehn) Schritte durch direktes Anklicken und anschließendes Ziehen mit der Maus verschiebbar.
		\item Mehrere (mindestens zwei) selektierte Kanten sind um einen Schritt durch direktes Anklicken und anschließendes Ziehen mit der Maus verschiebbar.
		\item Mehrere (mindestens zwei) selektierte Kanten sind um mehrere (mindestens zehn) Schritte durch direktes Anklicken und anschließendes Ziehen mit der Maus verschiebbar.
		
		
		% Navigationspfeile
		\item Jede einzelne Kante des Startknotens ist in jede Richtung des dreidimensionalen Raumes um einen Schritt durch Anklicken der Navigationspfeile und anschließendes Ziehen mit der Maus verschiebbar.
		\item Jede einzelne Kante des Startknotens ist in jede Richtung des dreidimensionalen Raumes um mehrere (mindestens zehn) Schritte durch Anklicken der Navigationspfeile und anschließendes Ziehen mit der Maus verschiebbar.
		\item Mehrere (mindestens zwei) selektierte Kanten sind um einen Schritt durch Anklicken der Navigationspfeile und anschließendes Ziehen mit der Maus verschiebbar.
		\item Mehrere (mindestens zwei) selektierte Kanten sind um mehrere (mindestens zehn) Schritte durch Anklicken der Navigationspfeile und anschließendes Ziehen mit der Maus verschiebbar.
		
		% Nachbaubar.
		\item Der in \ref{} abgebildete, Knoten \glqq Schlaufe\grqq~ist erstellbar.
		\item Der in \ref{} abgebildete, Knoten \glqq Überleger\grqq~ist erstellbar.
		
		% Zurücksetzen.
		\item Jede einzelne Kante des Startknotens lässt sich nach ihrer Verschiebung in die vorige Position durch direktes Anklicken und anschließendes Ziehen zurücksetzen.
		\item Jede einzelne Kante des Startknotens lässt sich nach ihrer Verschiebung in die vorige Position durch Anklicken des \glqq Undo\grqq-Buttons zurücksetzen.
		\item Jede einzelne Kante des Startknotens lässt sich nach ihrer Verschiebung in die vorige Position durch Anklicken des \glqq Undo\grqq-Buttons zurücksetzen und der \glqq Redo\grqq-Button macht die Aktion des \glqq Undo\grqq-Buttons rückgängig.
		\\
	
	\end{enumerate}
	
	% Ausgehend von einem zufällig erstellten Knoten ...
	
	\item[FT\_020] \textit{Eine neue Challenge aus zwei im Creative-Mode erzeugten Knoten erstellen.} \hfill\\
	
	\begin{enumerate}
	
		\item Im Hauptmenü auf den Text \glqq NEW Creative\grqq~klicken.
		\item Im folgenden Menü auf den Text \glqq NEW Challenge\grqq~klicken.
		\item Im folgenden Menü in der linken Auswahlliste einen Zielknoten
		\item In der rechten Liste einen Startknoten auswählen.
		\item Im rechteckigen Eingabefeld einen Namen für die Challenge eingeben und mit bestätigen.
		
	\end{enumerate}


\end{description}









