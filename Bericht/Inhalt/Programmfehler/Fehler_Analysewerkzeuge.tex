% !TeX encoding = UTF-8
%



\section{Werkzeuge}
\label{Abschnitt:Programmfehler:Werkzeuge}

Bei der Fehlersuche unterstützen uns mehrere Programme.
Je nach Fehlerklasse (s. \ref{Abschnitt:Programmfehler:Uebersicht:Klassifizierung}) sind verschiedene Werkzeuge hilfreich.





\subsection{Manuelle Unterst{"u}tzung}
\label{Abschnitt:Programmfehler:Werkzeuge:Manuell}


\begin{description}

	\item[FastStone Capture], \textit{V. 7.7}\hfill
		\\
		\\
		FastStone Capture erstellt Bildschirmaufnahmen. Damit lassen sich Screenshots und Videos von mehreren Fenstern machen. In den Videos werden auch Benutzerinteraktionen eingezeichnet. Gerade bei Fehlern, die sich durch grafisches Fehlverhalten äußern und um diese zu dokumentieren, kommt dieses Werkzeug zum Einsatz. Die Screenshots helfen bei der Fehlerbeschreibung. Zudem lassen sich die Videos - deren Größe wenige MB beträgt - einfach ins GIF-Format konvertieren. Das ist besonders hilfreich, da sich bis jetzt unter GitHub nur GIF-Animationen in die textuelle Beschreibung direkt einfügen lassen. Im Gegensatz zu den anderen Werkzeugen ist diese Software Shareware.
		
		\begin{tabbing}
			Internetseite:
			\= ~ \href {http://www.faststone.org}
		    	       {http://www.faststone.org}
		    \\
		\end{tabbing}
		
		\item[GitHub-Ticket-System] \hfill
		\\
		\\
		Das durch GitHub bereit gestellte Ticket-System nutzen wir zur Fehlerverfolgung. Sämtliche Probleme sind unter
		
		\begin{tabbing}
				\= ~ \href {https://github.com/pse-knot/pse-knot/issues}
    	       			   {https://github.com/pse-knot/pse-knot/issues}
    	       			   
		\end{tabbing} aufgelistet.
		

\end{description}




\subsection{Automatisierte {"U}berpr{"u}fung}
\label{Abschnitt:Programmfehler:Werkzeuge:Automatisiert}







