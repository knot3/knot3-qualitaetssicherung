% !TeX encoding = UTF-8
%



\chapter{Schluss}
\label{Kapitel:Abschluss}

~\\



\section{Bewertung}
\label{Abschnitt:Abschluss:Bewertung}

Die Qualität des Spiels Knot3 ist verbessert worden. Durch eine Komponententestabdeckung von über 80 \% und Funktionstests für alle elementaren Spielfunktionen ist jetzt die einwandfreie Funktion von Knot3 gewährleistet.\\

Alle Grundfunktionen sind getestet und funktionieren zuverlässig. Bei der Abnahme in Spielbarkeitstests durch menschliche Testspieler hat sich gezeigt, dass unser Produkt, wenn auch nicht perfekt, für eine interessante Spielerfahrung geeignet und für den Einsatz bereit ist.\\

In den Negativtests sichern wir das Programm weiter ab. Das Spiel ist so robust gegen die wahrscheinlichsten Störfälle. Die Algorithmen für die Knotentransformationen erfüllen ihre Aufgaben und es ist nicht möglich ungültige Knoten zu erzeugen (NT\_10, NT\_30, Komponententests). Die Robustheit wird zudem in Extremtests bei der Verarbeitung großer Datenmengen sichergestellt, so dass es bei realitätsnahen Knotengrößen mit bis zu 1000 Kanten zu keinen Problemen kommt (ET\_1). Erst bei 5000 Kanten verringert sich die Spielgeschwindigkeit gravierend. Die Hardware der Testsysteme ist schon etwas älter. D.h. Knot3 läuft auf allen gängigen Systemen. Des weiteren ist durch Tests überprüft worden, dass bei nicht zugewiesenen Eingaben kein Fehlverhalten auftritt (NT\_50). Der Spieler wird während dem gesamten Spiel dadurch unterstützt, dass wir ungültige Aktionen nicht zulassen.\\

Selbst bei Systemfehlern oder unerwarteten Fehlern geben wir einen Hinweis in Form einer Textmeldung an den Spieler. Diese Nachrichten sind im Nachhinein verwendbar, falls wir trotz allem etwas übersehen haben sollten, um den Fehler schnell zu finden und eine Aktualisierung bereit zu stellen.\\

Ein weiterer Grund, der nicht nur für die Funktionstüchtigkeit spricht ist der erfolgreiche Einsatz von automatischen Werkzeugen bei der Fehlersuche. Gendarme hat mit zu einer besseren Code-Qualität verbessert und insgesamt auf mehr als 1000 Probleme hingewiesen. 75 \% davon haben wir korrigiert. Darunter befanden sich auch zahlreiche kleinere Verbesserungen der Performance. Die restlichen 25 \% sind nicht weiter schlimm. Denn Gendarme zeigt neben ernstzunehmenden Problemen oft kleinere syntaktische Feinheiten. Wir nutzen diese Zeit besser für die Lösung der Probleme die wir bei unserer eigenen Fehlersuche gefunden haben. 








