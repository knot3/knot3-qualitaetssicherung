% !TeX encoding = UTF-8
%



\clearpage



\phantomsection
\label{Anhang:Anzeigefehler}



\subsection*{Anzeigefehler}



\begin{description}


\item[\#82] \hfill \\
Es ist immer noch möglich \glqq fehlende Modelle im Knoten zu erzeugen\grqq, \hyperref[Anhang:Grafikfehler:Loechrige_Uebergaenge]{s. Abb. A.5, auf S. \pageref{Anhang:Grafikfehler:Loechrige_Uebergaenge}}.

{\bfseries Lösung:} Skalierung der Modelle wurde sichergestellt.


\item[\#92] \hfill \\
Fehlerhafte Darstellung bei Übergängen. Wenn sich die rohrartige Geometrie in einer Gitterkreuzung trifft, sind je nach Perspektive z.B. Überlappungen wahrnehmbar, \hyperref[Anhang:Grafikfehler:Verschmolzene_Geometrie]{s. Abb. A.4, auf S. \pageref{Anhang:Grafikfehler:Verschmolzene_Geometrie}}.

{\bfseries Partielle Lösung:} Neue Modelle und neue Skalierungen wurden eingesetzt.


\item[\#96] \hfill \\
\glqq Start\grqq-Schaltfläche ist inkonsistent zum restlichen Design

{\bfseries Lösung:} Fehlende Linien wurden hinzugefügt.


\item[\#101] \hfill \\
Redo-Button bei Challenge-Start.\\
Redo-Butten von Anfang an sichtbar.
 
{\bfseries Lösung:} Sichtbarkeit zu Beginn auf \glqq false\grqq.


\item[\#119] \hfill \\
Verschieben des Pause-Menüs. \\
Die Ränder des Pause-Menüs verschieben sich nicht.

{\bfseries Lösung:} Ränder werden dynamisch positioniert. 


\item[\#108] \hfill \\
Challenges: Highscores und Menüeinträge nicht getrennt voneinander/wahrnehmbar.
 
{\bfseries Lösung:} Menüeinträge sind nun am unteren Rand des Dialogs.
 
 
\end{description}



~\\


