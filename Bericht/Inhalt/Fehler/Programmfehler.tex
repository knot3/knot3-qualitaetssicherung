% !TeX encoding = UTF-8
%



\subsection*{Programmfehler}


~\\
\begin{description}

\item[\#102] \hfill \\
Unter dem Creative-\glqq~Save As\grqq-Dialog sind Knotentransformationen u. weitere Eingaben möglich.

{\bfseries Lösung:} Gamescreen ignoriert Eingaben bei geöffnetem Dialog. Dies wurde durch das Hinzufügen einer Boolean für modale Dialoge erreicht. 


\item[\#95] \hfill \\
Mehrere Pause-Menüs (auch übereinander) öffnen.
Esc schließt nicht aktuellen Dialog, sondern öffnet neues Pause-Menü. \ref{Anhang:Grafikfehler:Dialog_Pause}

{\bfseries Lösung:} Die Zeichenreihenfolge wurde angepasst, sodass der aktuelle Dialog Esc abfängt.


\item[\#93] \hfill \\
Erstellen einer Challenge aus zwei gleichen Knoten

{\bfseries Lösung:}  Es wird vor dem Erstellen der Challenge auf Knotengleichheit geprüft, sodass keine ungültigen Challenges mehr erstellt werden können.


\item[\#97] \hfill \\
Beim Laden eines Knotens, bei einem zweiten Klick auf den Knoten verschwindet die Knoteninfo.

{\bfseries Lösung:} Knoteninfo wird nicht mehr pauschal gelöscht, sondern nur wenn sie sich ändert. 


\item[\#117] \hfill \\
Einflussbereich für Musikeinstellung. \\
Regler reagiert auf Klicks im gesammten Screen-Bereich.

{\bfseries Lösung:} Widget prüft ob Maus Regler bewegt.


\item[\#83] \hfill \\
Knoten nach Außerhalb des umgebenden Würfels bauen. \\
Kannten können aus der Skybox heraus gezogen werden.

{\bfseries Lösung:} Die Skybox vergrößert sich nun automatisch.


\item[\#107] \hfill \\

\glqq Redo/Undo\grqq~nach bestandener Challenge immer noch interaktiv. 

{\bfseries Lösung:} Button-Sichtbarkeit wird beim Beenden der Challenge auf 
\glqq false\grqq~gesetzt.

\item[\#84] \hfill \\

\glqq Überzoomen\grqq, sehr nahes ranzoomen ist problematisch, flippt manchmal die Kamera.

{\bfseries Lösung:} Zoomen begrenzt auf einen Minimalwert.


\item[\#103] \hfill \\
Eingabe von Whitespace (z.B. Leerzeichen, eins oder mehrere) dort wo Strings vom Spieler festgelegt werden. 

{\bfseries Lösung:} Whitespace-only Eingaben werden nun nicht mehr erlaubt.


\item[\#105] \hfill \\
Tastaturbelegung: Festlegen der gleichen Taste für mehrere Aktionen. 

{\bfseries Lösung:} Nun kann eine Taste nur einer Aktion zugewiesen werden.


\item[\#147] \hfill \\
Spielbarkeit: Knotentransformationen, Übergänge, Kamera. 

{\bfseries Lösung:} Nun kann eine Taste nur einer Aktion zugewiesen werden.


\end{description}

~\\


