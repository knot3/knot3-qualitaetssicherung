% !TeX encoding = UTF-8
%



\chapter{Einleitung}
\label{Kapitel:Einleitung}

In diesem Dokument beschreiben wir, was in der Qualitätssicherungsphase von uns erarbeitet wurde. Aus diesen Daten ziehen wir Rückschlüsse, welche qualitativen Anforderungen wir gezielt untersucht und getestet haben und geben eine Bewertung der Qualität des Spiels an.\\

Da in der Implementierungsphase aus zeitlichen Gründen nicht alle Pflichtenheft-Spezifikationen umgesetzt wurden, haben wir Ausstehendes in der QS-Phase nachgearbeitet. U. A. die Knoten-Vorschau beim Transformieren gehört zu den Wichtigeren Nachträgen. Änderungen und Ergänzungen, welche nicht oder in den vorigen Phasen anders spezifiziert waren, werden in diesem Bericht dokumentiert. Das betrifft z.B. zusätzliches Text-Material, welches wir zusammen mit dem Spiel ausliefern, um das Spiel an sich, die Steuerung oder Inhalte für die Spieler zu beschreiben. Neben den Angaben zu den durchgeführten Tätigkeiten und Änderungen enthält der Bericht auch die Dinge, deren Umsetzung nicht möglich war.\\

Zur Unterstützung der Zusammenarbeit und als Kommunikationsmittel verwendeten wir das \glqq Issues\grqq-Ticket-System. Dort sind alle von uns gefundenen Fehler erfasst. Deren Korrektur stand neben dem Testen in dieser Phase im Vordergrund. Der QS-Bericht enthält Protokolle zu den Fehlern, Tests und Änderungen, die im verbliebenen zeitlichen Rahmen möglich waren.\\

Die neueste Version von Knot3 mit allen Änderungen die während der Qualitätssicherung noch mit einflossen ist unter der Internetadresse\\

\begin{center}
\href{http://www.knot3.de/download/}{\mousecursor~http://www.knot3.de/download/}\\
\end{center}

oder auf GitHub unter\\

\begin{center}
\href{https://github.com/pse-knot/knot3-code/releases}{\mousecursor~https://github.com/pse-knot/knot3-code/releases}\\
\end{center}

verfügbar.



