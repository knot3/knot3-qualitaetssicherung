% !TeX encoding = UTF-8
%


% TODO: überarbeiten. 

\chapter{Einleitung}
\label{Kapitel:Einleitung}

In diesem Dokument beschreiben wir, was in der Qualitätssicherungsphase von uns erarbeitet wurde.
Wir haben die Qualität des Spiels verbessert, indem wir auf bestimmte Kriterien zur Qualitätssicherung Wert gelegt haben. Wir verwendeten zum Beispiel Funktionstest und Überdeckungstest, um die Qualität sicher zu stellen.\\~

Einige Gedanken, die wir uns im Laufe der Qualitätssicherungsphase zu dem Design gemacht haben, konnten aufgrund von Zeitmangel nicht umgesetzt werden. Sie werden in diesem Bericht aber auch erwähnt und beschrieben.\\~

Des Weiteren gab es Dinge die wir im Pflichtenheft spezifiziert haben, jedoch in der Implementierungsphase zeitlich nicht umsetzen konnten. Diese wurden von uns in der Qualitätssicherungsphase umgesetzt. So gibt es beispielsweise nun eine Knoten-Vorschau beim Transformieren.\\~

Änderungen und Ergänzungen, welche in den vorherigen Phasen nicht so vorgesehen waren, werden in diesem Bericht auch vorgestellt. So war beispielsweise noch nicht klar, wie genau man dem Spieler die grundlegenden Spielmechaniken erklären sollte. Aufgrund dessen haben wir eine separate PDF-Datei (\glqq Spielanleitung.pdf \grqq) geschrieben, welche den Tutorial-Mode ersetzen soll.\\~

Um ein bestmögliches Arbeiten des Teams zu gewährleisten nutzten wir das Ticket-System von Github. In das Ticket-System kann man alle wichtigen Aufgaben und Bugs eintragen und diese einem Teammitglied zuweisen. Dies hat den Vorteil, dass man eine Auflistung zu allen noch zu bearbeitenden Aufgaben hat und somit nichts vergessen kann. Es ermöglichte auch eine einfache und genau Übersicht über die Arbeitsverteilung innerhalb der Phase.
 











