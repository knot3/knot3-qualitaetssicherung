% !TeX encoding = UTF-8
%


% TODO: überarbeiten. 

\chapter{Einleitung}
\label{Kapitel:Einleitung}

In diesem Dokument beschreiben wir, was in der Qualitätssicherungsphase von uns erarbeitet wurde. % gewisse Qualität (?)
Es wurde versucht eine gewisse Qualität des Spiels zu gewährleisten, indem wir auf bestimmte Kriterien zur Qualitätssicherung Wert gelegt haben. Dementsprechend wollten wir beispielsweise, dass unser Spiel eine gewisse Zuverlässigkeit hat und somit nicht jedes mal abstürzt. Dies konnten wir unter anderem mit Funktionstest überprüfen und infolgedessen verbessern.\\~ % Bewertungen hier nicht, das gehört in den Abschlussteil (Pascal)

Einige Gedanken, die wir uns im Laufe der Qualitätssicherungsphase zu dem Design gemacht haben, konnten aufgrund von Zeitmangel nicht integriert werden. Sie werden in diesem Bericht auch erwähnt und beschrieben.\\~

% Ausstehendes ...

Änderungen und Ergänzungen, welche in den vorherigen Phasen nicht so vorgesehen waren, werden in diesem Bericht auch dargestellt. So war beispielsweise noch nicht klar, wie genau man dem Spieler die grundlegenden %Satz
 Spielmechaniken erklären sollte. Aufgrund dessen haben wir eine separate PDF-Datei (\glqq Spielanleitung.pdf \grqq) geschrieben, welche den Tutorial-Mode ersetzen soll.\\~

Um ein bestmögliches Arbeiten des Teams zu gewährleisten nutzten wir das Ticket-System von Github. In das Ticket-System kann man alle wichtigen Aufgaben und Bugs eintragen und diese einem Teammitglied zuweisen. Dies hat den Vorteil, dass man eine Auflistung zu allen noch zu bearbeitenden Aufgaben hat und somit nichts vergessen kann. 
 











