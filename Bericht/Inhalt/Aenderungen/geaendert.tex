% !TeX encoding = UTF-8
%



\subsection{Ge{"a}ndert}
\label{Abschnitt:Aenderungen:Protokoll:Behobene_Probleme}

Fehler die wir verbessert haben oder Ergänzungen werden hier beschrieben. Kleinigkeiten fließen nicht in das Protokoll ein.\\

% TODO: Was sind Kleinigkeiten.
% Eineitungssatz zu Änderungen ...




\subsubsection*{Startmenü, Credits}
\label{Abschnitt:Aenderungen:Protokoll:Startmenue}


\hyperref[Abb:Aenderungen:Startmenue_(vorher)]{vorher~(\mousecursor)}





\subsubsection*{Transformations-Vorschau}

% TODO







\subsubsection*{Spielbeschreibung}

% TODO










\subsubsection*{Tastenabfrage}
\textbf{Problem:}
Wir haben festgestellt, dass unser gewähltes Abtast-Intervall von 100 Millisekunden zu kurz ist für normale Tasteneingaben. Bei normalen Tippen konnte es passieren das nun die Taste zweimal abgetastet wurde. Was eingaben von Spielernamen erschwerte. \\
\textbf{Änderung:} Das Intervall wurde auf 100 Millisekunden erhöht um dieses Problem zu verhindern.

\subsection{Language}
\textbf{Problem:}
Ein Language Objekt aktualisierte nicht seine Attribute.
\\
\textbf{Änderung:}Die Attribute werden nun bei Bedarf direkt aus der Datei ausgelesen.

\subsection{Farbcodierung in Level-Dateiformat}
\textbf{Problem:}
Es konnten keine Farben in RGB angegeben werden. Es wurden immer RGBA werte erwartet.
Falls es nur 6 Hexadezimal-Zahlen (RGB) waren wurde eine Exception ausgelöst.
\\
\textbf{Änderung:} Es wird nun überprüft ob es sich um RGB oder RGBA handelt.

\subsection{Große IF-ELSE-Blöcke}
\textbf{Problem:}
In der Dekodierung und Kodierung der Richtung beim Laden bzw. Speichern von Knoten in das Dateiformat wurde die Zuordnung jeweils durch einen großen IF-ELSE-Block erledigt.
\\
\textbf{Änderung:} Es gibt nun ein Dictonary mit den Zuordnungen welches für beide Funktionen Verwendung findet.


% \subsubsection*{}







