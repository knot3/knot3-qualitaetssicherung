% !TeX encoding = UTF-8
%



\newpage



\subsection{Nicht Versch{"o}nertes} % Juice it up!
\label{Abschnitt:Aenderungen:Protokoll:Verschoenerungen:Nicht}

Während der QS-Phase haben wir nach dem Motto \glqq Juice It Up\grqq~Möglichkeiten geprüft, das Knot3-Spiel zu verschönern. Da die Zeit jedoch größtenteils zur Fehlerkorrektur genutzt wurde, konnten diese nicht mehr umgesetzt werden. Dennoch möchten wir die genannten Vorschläge hier dokumentieren.\\


\subsubsection*{Blinkende Sterne}
\label{Abschnitt:Aenderungen:Protokoll:Verschoenerungen:Nicht:Blinkende_Sterne}

Die Umgebung im Creative- oder im Challenge-Mode ist recht neutral. Wir haben eine spacige an den Weltraum oder einen Sternenhimmel erinnernde Skybox. Außer die vom Spieler initiierten Knotentransformationen gibt es keine grafischen Änderungen. Als einfache Erweiterung wurde angedacht, einige der bereits vorhandenen Sterne im Hintergrund zum Blinken zu animieren. Der Vorteil unserer Implementierung ist allerdings, dass der Spieler durch nichts abgelenkt wird.\\

\subsubsection*{Men{"u}f{"u}hrung und Men{"u}-Stil}
\label{Abschnitt:Aenderungen:Protokoll:Verschoenerungen:Nicht:Menues}

Im Grunde wollten wir im Menü (s. \ref{Abb:Aenderungen:Startmenue_(vorher)}) noch eine Möglichkeit bieten, z.B. durch eine Knotenschaltfläche im Hauptmenü, sich die Credits anzeigen zu lassen. Diese Änderungen ist schwierig, da die Erstellung der grafischen Oberflächen bei unserer Implementierung mit vielen hart-kodierten Werten viel Zeit beansprucht. Das Einfügen eines Credits-Buttons verschöbe die bereits vorhandenen Komponenten und erfordert ein Neuentwurf des Hauptmenüs. 

Beim Durchspielen dieser Idee haben wir gleichzeitig geprüft wie hoch der Aufwand für eine optisch ansprechendere Darstellung der Menüs ist. Wir haben dazu ein Mock-Up für das Hauptmenü erstellt, wie eine solche Änderung aussehen könnte, siehe Abb.\\







