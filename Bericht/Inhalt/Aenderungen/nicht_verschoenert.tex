% !TeX encoding = UTF-8
%



\newpage



\subsection{Nicht versch{"o}nert} % Juice it up!
\label{Abschnitt:Aenderungen:Protokoll:Verschoenerungen:Nicht}

Während der QS-Phase haben wir nach dem Motto \glqq Juice It Up\grqq~Möglichkeiten geprüft, das Knot3-Spiel zu verschönern. Da die Zeit jedoch größtenteils zur Fehlerkorrektur genutzt wurde, konnten diese nicht mehr umgesetzt werden. Dennoch möchten wir die genannten Vorschläge hier dokumentieren.\\


\subsubsection*{Blinkende Sterne}
\label{Abschnitt:Aenderungen:Protokoll:Verschoenerungen:Nicht:Blinkende_Sterne}

Die Umgebung im Creative- oder im Challenge-Mode ist recht neutral. Wir haben eine spacige an den Weltraum oder einen Sternenhimmel erinnernde Skybox. Außer die vom Spieler initiierten Knotentransformationen gibt es keine grafischen Änderungen. Als einfache Erweiterung wurde angedacht, einige der bereits vorhandenen Sterne im Hintergrund zum Blinken zu animieren. Der Vorteil unserer Implementierung ist allerdings, dass der Spieler durch nichts abgelenkt wird.\\

\subsubsection*{Men{"u}f{"u}hrung und Men{"u}-Stil}
\label{Abschnitt:Aenderungen:Protokoll:Verschoenerungen:Nicht:Menues}

Im Hauptmenü, \hyperref[Abschnitt:Anhang:Aenderungen:Nicht]{ \mousecursor~s. Anhang, S. \pageref{Abschnitt:Anhang:Aenderungen:Nicht}}, wollten wir durch eine Knotenschaltfläche noch die Möglichkeit bieten, sich die Mitwirkenden anzeigen zu lassen. Diese Änderungen ist zu aufwendig, da die Erstellung der grafischen Oberflächen bei unserer Implementierung mit vielen hart-kodierten Werten viel Zeit beansprucht. Das Einfügen eines Credits-Buttons verschöbe die bereits vorhandenen Komponenten und erfordert so einen Neuentwurf des ganzen Hauptmenüs.\\

Mit dieser Überlegung haben wir gleichzeitig geprüft, wie hoch der Aufwand für eine optisch ansprechendere Darstellung der Menüs wäre. Wir haben dazu ein Mock-Up für das Hauptmenü erstellt, \hyperref[Abschnitt:Anhang:Aenderungen:Nicht]{ \mousecursor~s. Anhang, S. \pageref{Abschnitt:Anhang:Aenderungen:Nicht}}.\\







