% !TeX encoding = UTF-8
%
% LaTeX-Pakete.
%


%
% Neue Deutsche Rechtschreibung.
%
\usepackage[ngerman]{babel}

%
% Automatische Kodierung vieler Sonderzeichen.
%
\usepackage[utf8x]{inputenc}

%
% Z.B. Symbol: Schwarzer Stern.
%
\usepackage{amssymb}

%
%
%
%\usepackage{tikz}

%
%
%
\usepackage[T1]{fontenc}

%
% Schriftpaket für "Latin Modern"
%
\usepackage{lmodern}

%
% Schriftfarbe.
%
\usepackage{xcolor}

%
% Tabellen über mehrere Seiten.
%
\renewcommand*{\arraystretch}{1.8} % Zeilengesamthöhe.
\usepackage{longtable}
\usepackage{multirow}

%
% Definition spezieller Zeilentypen z.B. für die longtable-Umgebung.
%
\usepackage{array}
\newcolumntype{L}[1]{>{\raggedright\let\newline\\\arraybackslash\hspace{0pt}}m{#1}}
\newcolumntype{C}[1]{>{\centering\let\newline\\\arraybackslash\hspace{0pt}}m{#1}}
\newcolumntype{R}[1]{>{\raggedleft\let\newline\\\arraybackslash\hspace{0pt}}m{#1}}



%
%
%
\usepackage[justification=RaggedRight,singlelinecheck=off]{caption}



%
% Abstand links, rechts, vor, nach Abschnitten (engl. "section")
%
\usepackage{titlesec}
\titlespacing{\section}{0pt}{5.5ex plus 1ex minus .2ex}{4.3ex plus .2ex}
\titlespacing*{\subsection}{0pt}{5.5ex plus 1ex minus .2ex}{4.3ex plus .2ex}

%
% Abstände.
%
\usepackage{vmargin}
\setpapersize{A4} % besser oben?
\setmarginsrb{3cm}{1.5cm}{3cm}{2cm}{6mm}{6mm}{5mm}{15mm}

%
%
%
%\usepackage[pdftex]{graphicx}
\usepackage{graphicx}

%
% Skalierbare Vektorgrafiken (SVG'en) einbinden.
%
% HINWEIS: Inkscape, pdflatex --shell-escape nötig.
%
\usepackage{svg}

%
% Inkludieren von PDF-Dokumenten.
%
%\usepackage{pdfpages}

%
% Verlinkungen verfügbar und "klickbar" machen.
%
\usepackage[hyphens]{url}
\usepackage[hidelinks]{hyperref} % hidelinks

%
% Glossare.
%
% HINWEIS: perl nötig.
%
\usepackage[nomain]{glossaries}
% \newglossary{FA}{fai}{fao}{Fachausdrücke}
% \newglossary{AK}{aki}{ako}{Abkürzungen}
\makeglossaries
%% Glossar-Einträge im Voraus bekannt machen,
%% damit sie im Folgenden verwendbar sind.
% \loadglsentries[FA]{Inhalt/Verzeichnisse/Glossare/Fachausdruecke}
% \loadglsentries[AK]{Inhalt/Verzeichnisse/Glossare/Abkuerzungen}
%
% HINWEIS:
%
% Begriffe, die im Glossar auftauchen müssen auch im
% Text auftauchen und durch gls{Begriff} markiert worden
% sein, damit sie im Verzeichnis gelistet werden!
%

