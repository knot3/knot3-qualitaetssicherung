\documentclass{report}
\usepackage[utf8]{inputenc}
\usepackage[ngerman]{babel}
\usepackage{microtype}
\usepackage{enumitem}
\usepackage{graphicx}

\begin{document}

\title{\textbf{Spielanleitung für KNOT$^3$}}
\date{\today}

\maketitle

\section*{Einleitung}
Knot$^3$ ist ein Knobel- und Konstruktionsspiel mit zwei Spielmodi.
Im Creative-Mode kann man seiner Kreativität freien Lauf lassen und aus einem einfachen quadratischen Knoten einen komplexen Knoten konstruieren.\\
Das Ziel des Challenge-Mode ist es den Ausgangsknoten (Knoten auf der rechten Bildschirmseite) so schnell wie möglich in den Referenzknoten (Knoten auf der linken Bildschirmseite) zu überführen. \\~\\
Der folgende Überblick über die möglichen Spielinteraktionen wird anhand der Standarttastaturbelegung beschrieben. Im Einstellungsmenü kann der Spieler jedoch die Tastaturbelegung an seinen Vorlieben anpassen.\\

\section*{Mögliche Interaktionen mit dem Knoten}
Der Spieler kann eine Kante des Knotens in die gewünschte Richtung verschieben, indem er auf der Kante die linke Maustaste gedrückt hält und die Kante dann in die gewünschte Richtung verschiebt. Alternativ kann der Spieler auch Pfeile im Einstellungsmenü einschalten,  welche er zum Verschieben auch anklicken kann. Es besteht auch die Möglichkeit mehrere Kanten auszuwählen und zu verschieben. Zum Auswählen muss der Spieler die linke Control-Taste gedrückt halten und kann dann mit der linken Maustaste alle Kanten auswählen die er verschieben möchte. Mit Hilfe der linken Shift-Taste kann man auch mehrere aneinander liegende Kanten zur Auswahl hinzufügen.\\~\\
Um eine oder mehrere Kanten einzufärben muss man diese auswählen und dann die C Taste drücken. Dabei öffnet sich ein Fenster mit einer Liste an Farben, welche man auswählen kann. Das Einfärben von Kanten funktioniert nur im Creative-Mode.

\section*{Kamerabewegungen}
Mit den WASD Tasten kann der Spieler die Kamera nach oben/unten/rechts/links bewegen. Mit Hilfe der Tasten R und F kann der Spieler die Kamera in der Ebene nach vorne und nach hinten bewegen.\\
Um einen besseren Überblick über seinen Knoten zu erhalten kann der Spieler mit den Tasten Q und E (alternativ mit dem Mausrad) herein- und herauszoomen.\\
Möchte der Spieler seinen Knoten aus einem anderen Blickwinkel betrachten, so kann er die Kamera auch um eine Kante des Knotens rotieren. Dafür muss der Spieler eine Kante mit der rechten Maustaste auswählen und mit Hilfe der Pfeiltasten (alternativ durch gedrückt halten der rechten Maustaste) kann er daraufhin um die Kante rotieren.\\
Mittels der Space-Taste springt der Mittelpunkt der Kamera auf den Mittelpunkt der selektierten Kante.\\
Möchte der Spieler sich von seiner Position aus einen Überblick über den Knoten verschaffen, so kann er mit der linken Alt-Taste die Kamera frei geben und sich mit der Maus umschauen. Durch erneutes Klicken der linken Alt-Taste rastet die Kamera wieder ein. Hat der Spieler die Kamera unglücklich verdreht, so kann er die Kamera mit Hilfe der Enter-Taste wieder zurücksetzen.




\end{document}
